\chapter*{Appendix}
\section{The Epistle to the Hebrews, Chapter 13 (NSRV~-~CNV)}
\footnotesize
\begin{tabularx}{\textwidth}{p{0.95\textwidth}}
\hline
Hebrews Ch. 1 \\
\hline \\
In many and various ways God spoke of old to our fathers by the prophets; \\
  神在古時候,曾經多次用種種方法,藉著先知向我們的祖先說話; \\ \\
but in these last days he has spoken to us by a Son, whom he appointed the heir of all things, through whom also he created the world. \\
 在這末後的日子,卻藉著他的兒子向我們說話。 神已經立他作萬有的承受者,並且藉著他創造了宇宙(“宇宙”或譯:“諸世界”或“眾世代”)。 \\ \\
He reflects the glory of God and bears the very stamp of his nature, upholding the universe by his word of power. When he had made purification for sins, he sat down at the right hand of the Majesty on high, \\
 他是 神榮耀的光輝,是 神本質的真象,用自己帶有能力的話掌管萬有;他作成了潔淨罪惡的事,就坐在高天至尊者的右邊。 \\ \\
having become as much superior to angels as the name he has obtained is more excellent than theirs. \\
 他所承受的名比天使的名更尊貴,所以他遠比天使崇高。 \\ \\
For to what angel did God ever say, "Thou art my Son, today I have begotten thee"? Or again, "I will be to him a father, and he shall be to me a son"? \\
  神曾對哪一個天使說過:“你是我的兒子,我今日生了你”呢?或者說:“我要作他的父親,他要作我的兒子”呢? \\ \\
And again, when he brings the first-born into the world, he says, "Let all God's angels worship him." \\
  神差遣長子到世上來的時候,又說:“ 神所有的天使都要拜他。” \\ \\
Of the angels he says, "Who makes his angels winds, and his servants flames of fire." \\
 論到天使,說:“ 神以他的天使為風,以他的僕役為火燄。” \\ \\
But of the Son he says, "Thy throne, O God, is for ever and ever, the righteous scepter is the scepter of thy kingdom. \\
 但是論到兒子,卻說:“ 神啊!你的寶座是永永遠遠的,你國的權杖,是公平的權杖。 \\ \\
Thou hast loved righteousness and hated lawlessness; therefore God, thy God, has anointed thee with the oil of gladness beyond thy comrades." \\
 你喜愛公義,恨惡不法,所以, 神,就是你的 神,用喜樂的油膏抹你,勝過膏抹你的同伴。” \\ \\
And, "Thou, Lord, didst found the earth in the beginning, and the heavens are the work of thy hands; \\
 又說:“主啊!你起初立了地的根基,天也是你手的工作。 \\ \\
they will perish, but thou remainest; they will all grow old like a garment, \\
 天地都要毀滅,你卻長存;天地都要像衣服一樣漸漸殘舊, \\ \\
like a mantle thou wilt roll them up, and they will be changed. But thou art the same, and thy years will never end." \\
 你要把天地像外套一樣捲起來,天地就像衣服一樣被更換;只有你永不改變,你的年數也沒有窮盡。” \\ \\
But to what angel has he ever said, "Sit at my right hand, till I make thy enemies a stool for thy feet"? \\
  神可曾向哪一個天使說:“你坐在我的右邊,等我使你的仇敵作你的腳凳”呢? \\ \\
Are they not all ministering spirits sent forth to serve, for the sake of those who are to obtain salvation? \\
 天使不都是服役的靈,奉差遣為那些要承受救恩的人效勞嗎? \\ \\
\hline
\end{tabularx}

\newpage
`
\begin{tabularx}{\textwidth}{p{0.95\textwidth}}
\hline
Hebrews Ch. 2 \\
\hline \\
Therefore we must pay the closer attention to what we have heard, lest we drift away from it. \\
因此,我們必須更加密切注意所聽過的道理,免得我們隨流失去。 \\ \\
For if the message declared by angels was valid and every transgression or disobedience received a just retribution, \\
那透過天使所傳講的信息既然是確定的,所有干犯和不聽從的,都受了應得的報應。 \\ \\
how shall we escape if we neglect such a great salvation? It was declared at first by the Lord, and it was attested to us by those who heard him, \\
如果我們忽略了這麼大的救恩,怎麼能逃罪呢?這救恩起初是由主親自宣講的,後來聽見的人給我們證實了。 \\ \\
while God also bore witness by signs and wonders and various miracles and by gifts of the Holy Spirit distributed according to his own will. \\
 神又照著自己的旨意,用神蹟、奇事和各樣異能,以及聖靈的恩賜,與他們一同作見證。 \\ \\
For it was not to angels that God subjected the world to come, of which we are speaking. \\
 神並沒有把我們所說的“將來的世界”,交給天使管轄; \\ \\
It has been testified somewhere, "What is man that thou art mindful of him, or the son of man, that thou carest for him? \\
但是有人在聖經上某一處證實說:“人算甚麼,你竟記念他?世人算甚麼,你竟眷顧他? \\ \\
Thou didst make him for a little while lower than the angels, thou hast crowned him with glory and honor, \\
你使他暫時成了比天使卑微(“暫時成了比天使卑微”或譯“比天使稍低微一點”),卻賜給他榮耀尊貴作冠冕,(有些古卷在此有“並立他統管你手所造的一切”一句) \\ \\
putting everything in subjection under his feet." Now in putting everything in subjection to him, he left nothing outside his control. As it is, we do not yet see everything in subjection to him. \\
使萬物都服在他的腳下。”既然萬有都服了他,就沒有剩下一樣不服他的了。但是現在我們還沒有看見萬有都服他。 \\ \\
But we see Jesus, who for a little while was made lower than the angels, crowned with glory and honor because of the suffering of death, so that by the grace of God he might taste death for every one. \\
不過,我們看見那位暫時成了比天使卑微(“暫時成了比天使卑微”或譯“比天使稍低微一點”的耶穌,因為受了死的痛苦,就得了榮耀尊貴作冠冕,好叫他因著 神的恩典,為萬人嘗了死味。 \\ \\
For it was fitting that he, for whom and by whom all things exist, in bringing many sons to glory, should make the pioneer of their salvation perfect through suffering. \\
萬有因他而有、藉他而造的那位,為了要帶領許多兒子進入榮耀裡去,使救他們的元首藉著受苦而得到成全,本是合適的。 \\ \\
For he who sanctifies and those who are sanctified have all one origin. That is why he is not ashamed to call them brethren, \\
因為那位使人成聖的,和那些得到成聖的,同是出於一個源頭;所以他稱他們為弟兄也不以為恥。 \\ \\
saying, "I will proclaim thy name to my brethren, in the midst of the congregation I will praise thee." \\
他說:“我要向我的弟兄宣揚你的名,我要在聚會中歌頌你。” \\ \\
And again, "I will put my trust in him." And again, "Here am I, and the children God has given me." \\
又說:“我要信靠他。”又說:“看哪,我和 神所賜給我的孩子們。” \\ \\
Since therefore the children share in flesh and blood, he himself likewise partook of the same nature, that through death he might destroy him who has the power of death, that is, the devil, \\
孩子們既然同有血肉之體,他自己也照樣成為血肉之體,為要藉著死,消滅那掌握死權的魔鬼, \\ \\
and deliver all those who through fear of death were subject to lifelong bondage. \\
並且要釋放那些因為怕死而終身作奴僕的人。 \\ \\
For surely it is not with angels that he is concerned but with the descendants of Abraham. \\
其實,他並沒有救援天使,只救援亞伯拉罕的後裔。 \\ \\
Therefore he had to be made like his brethren in every respect, so that he might become a merciful and faithful high priest in the service of God, to make expiation for the sins of the people. \\
所以,他必須在各方面和他的弟兄們相同,為了要在 神的事上,成為仁慈忠信的大祭司,好為人民贖罪。 \\ \\
For because he himself has suffered and been tempted, he is able to help those who are tempted. \\
因為他自己既然經過試探,受了苦,就能夠幫助那些被試探的人。 \\ \\
\hline
\end{tabularx}

\newpage
`
\begin{tabularx}{\textwidth}{p{0.95\textwidth}}
\hline
Hebrews Ch. 3 \\
\hline \\
Therefore, holy brethren, who share in a heavenly call, consider Jesus, the apostle and high priest of our confession. \\
因此,同蒙天召的聖潔的弟兄啊!你們應該想想耶穌,就是作使徒、作我們所宣認的大祭司的那一位。 \\ \\
He was faithful to him who appointed him, just as Moses also was faithful in God's house. \\
他忠於那位委派他的,好像摩西在 神的全家盡忠一樣。 \\ \\
Yet Jesus has been counted worthy of as much more glory than Moses as the builder of a house has more honor than the house. \\
他比摩西配得更大的榮耀,好像建造房屋的人比房屋配得更大的尊貴一樣。 \\ \\
(For every house is built by some one, but the builder of all things is God.) \\
因為每一間房屋都是人建造的,只有萬物是 神建造的。 \\ \\
Now Moses was faithful in all God's house as a servant, to testify to the things that were to be spoken later, \\
摩西不過是個僕人,在 神的全家盡忠,為以後要傳講的事作證; \\ \\
but Christ was faithful over God's house as a son. And we are his house if we hold fast our confidence and pride in our hope. \\
但是基督卻是兒子,管理自己的家。如果我們把坦然無懼的心和可誇的盼望持守到底,我們就是他的家了。 \\ \\
Therefore, as the Holy Spirit says, "Today, when you hear his voice, \\
所以,就像聖靈所說的:“如果你們今天聽從他的聲音, \\ \\
do not harden your hearts as in the rebellion, on the day of testing in the wilderness, \\
就不要硬著心,好像在曠野惹他發怒、試探他的日子一樣; \\ \\
where your fathers put me to the test and saw my works for forty years. \\
在那裡,你們的祖先以試驗來試探我,觀看我的作為有四十年之久。 \\ \\
Therefore I was provoked with that generation, and said, `They always go astray in their hearts; they have not known my ways.' \\
所以,我向那個世代的人發怒,說:‘他們心裡常常迷誤,不認識我的道路。’ \\ \\
As I swore in my wrath, `They shall never enter my rest.'" \\
我就在烈怒中起誓,說:‘他們絕不可進入我的安息。’” \\ \\
Take care, brethren, lest there be in any of you an evil, unbelieving heart, leading you to fall away from the living God. \\
弟兄們,你們要小心,免得你們中間有人存著邪惡、不信的心,以致離棄了永活的 神; \\ \\
But exhort one another every day, as long as it is called "today," that none of you may be hardened by the deceitfulness of sin. \\
趁著還有叫作“今天”的時候,總要天天互相勸勉,免得你們中間有人受了罪惡的誘惑,心裡就剛硬了。 \\ \\
For we share in Christ, if only we hold our first confidence firm to the end, \\
如果我們把起初的信念堅持到底,就是有分於基督的人了。 \\ \\
while it is said, "Today, when you hear his voice, do not harden your hearts as in the rebellion." \\
經上說:“如果你們今天聽從他的聲音,就不要硬著心,像惹他發怒的時候一樣。” \\ \\
Who were they that heard and yet were rebellious? Was it not all those who left Egypt under the leadership of Moses? \\
那時,聽了他的話而惹他發怒的是誰呢?不就是摩西從埃及領出來的那些人嗎? \\ \\
And with whom was he provoked forty years? Was it not with those who sinned, whose bodies fell in the wilderness? \\
四十年之久, 神向誰發怒呢?不就是向那些犯了罪陳屍曠野的人嗎? \\ \\
And to whom did he swear that they should never enter his rest, but to those who were disobedient? \\
他又向誰起誓說,他們絕對不可以進入他的安息呢?不就是向那些不順從的人嗎? \\ \\
So we see that they were unable to enter because of unbelief. \\
這樣看來,他們不能進入安息,是因為不信的緣故。 \\ \\

\hline
\end{tabularx}

\newpage
`
\begin{tabularx}{\textwidth}{p{0.95\textwidth}}
\hline
Hebrews Ch. 4 \\
\hline \\
Therefore, while the promise of entering his rest remains, let us fear lest any of you be judged to have failed to reach it. \\
所以,那進入安息的應許,既然還給我們留著,我們就應該戰戰兢兢,恐怕我們中間有人像是被淘汰了。 \\ \\
For good news came to us just as to them; but the message which they heard did not benefit them, because it did not meet with faith in the hearers. \\
因為有福音傳給我們,像傳給他們一樣,只是他們所聽見的道,對他們沒有益處,因為他們沒有用信心與所聽見的打成一片(“沒有用信心與所聽見的打成一片”,有古卷作“沒有用信心與聽從這道的人打成一片”)。 \\ \\
For we who have believed enter that rest, as he has said, "As I swore in my wrath, `They shall never enter my rest,'" although his works were finished from the foundation of the world. \\
然而我們信了的人,就可以進入那安息。正如 神所說:“我在烈怒中起誓說,他們絕不可進入我的安息!”其實 神的工作,從創立世界以來已經完成了。 \\ \\
For he has somewhere spoken of the seventh day in this way, "And God rested on the seventh day from all his works." \\
因為論到第七日,他在聖經某一處說:“在第七日 神歇了他的一切工作。” \\ \\
And again in this place he said, "They shall never enter my rest." \\
但在這裡又說:“他們絕不可進入我的安息。” \\ \\
Since therefore it remains for some to enter it, and those who formerly received the good news failed to enter because of disobedience, \\
既然這安息還留著要讓一些人進去,但那些以前聽過福音的人,因為不順從不得進去; \\ \\
again he sets a certain day, "Today," saying through David so long afterward, in the words already quoted, "Today, when you hear his voice, do not harden your hearts." \\
所以 神就再定一個日子,就是過了很久以後,藉著大衛所說的“今天”,就像前面引用過的:“如果你們今天聽從他的聲音,就不要硬著心。” \\ \\
For if Joshua had given them rest, God would not speak later of another day. \\
如果約書亞已經使他們享受了安息, 神後來就不會再提到別的日子了。 \\ \\
So then, there remains a sabbath rest for the people of God; \\
這樣看來,為了 神的子民,必定另外有一個“安息日”的安息保留下來。 \\ \\
for whoever enters God's rest also ceases from his labors as God did from his. \\
因為那進入 神安息的人,就歇了自己的工作,好像 神歇了自己的工作一樣。 \\ \\
Let us therefore strive to enter that rest, that no one fall by the same sort of disobedience. \\
所以,我們要竭力進入那安息,免得有人隨著那不順從的樣子就跌倒了。 \\ \\
For the word of God is living and active, sharper than any two-edged sword, piercing to the division of soul and spirit, of joints and marrow, and discerning the thoughts and intentions of the heart. \\
因為 神的道是活的,是有效的,比一切兩刃的劍更鋒利,甚至可以刺入剖開魂與靈,關節與骨髓,並且能夠辨明心中的思想和意念。 \\ \\
And before him no creature is hidden, but all are open and laid bare to the eyes of him with whom we have to do. \\
被造的在 神面前沒有一樣不是顯明的,萬有在他的眼前都是赤露敞開的;我們必須向他交帳。 \\ \\
Since then we have a great high priest who has passed through the heavens, Jesus, the Son of God, let us hold fast our confession. \\
我們既然有一位偉大的、經過了眾天的大祭司,就是 神的兒子耶穌,就應該堅持所宣認的信仰。 \\ \\
For we have not a high priest who is unable to sympathize with our weaknesses, but one who in every respect has been tempted as we are, yet without sin. \\
因為我們的大祭司並不是不能同情我們的軟弱,他像我們一樣,也曾在各方面受過試探,只是他沒有犯罪。 \\ \\
Let us then with confidence draw near to the throne of grace, that we may receive mercy and find grace to help in time of need. \\
所以,我們只管坦然無懼地來到施恩的寶座前,為的是要領受憐憫,得到恩惠,作為及時的幫助。 \\ \\

\hline
\end{tabularx}


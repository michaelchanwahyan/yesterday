\chapter*{Appendix}
\section*{The Epistle to the Hebrews (NSRV~-~CNV)}
\small
\begin{tabularx}{\textwidth}{p{0.95\textwidth}}
\hline
Hebrews Ch. 1 \\
\hline \\
In many and various ways God spoke of old to our fathers by the prophets; \\
  神在古時候,曾經多次用種種方法,藉著先知向我們的祖先說話; \\ \\
but in these last days he has spoken to us by a Son, whom he appointed the heir of all things, through whom also he created the world. \\
 在這末後的日子,卻藉著他的兒子向我們說話。 神已經立他作萬有的承受者,並且藉著他創造了宇宙(“宇宙”或譯:“諸世界”或“眾世代”)。 \\ \\
He reflects the glory of God and bears the very stamp of his nature, upholding the universe by his word of power. When he had made purification for sins, he sat down at the right hand of the Majesty on high, \\
 他是 神榮耀的光輝,是 神本質的真象,用自己帶有能力的話掌管萬有;他作成了潔淨罪惡的事,就坐在高天至尊者的右邊。 \\ \\
having become as much superior to angels as the name he has obtained is more excellent than theirs. \\
 他所承受的名比天使的名更尊貴,所以他遠比天使崇高。 \\ \\
For to what angel did God ever say, "Thou art my Son, today I have begotten thee"? Or again, "I will be to him a father, and he shall be to me a son"? \\
  神曾對哪一個天使說過:“你是我的兒子,我今日生了你”呢?或者說:“我要作他的父親,他要作我的兒子”呢? \\ \\
And again, when he brings the first-born into the world, he says, "Let all God's angels worship him." \\
  神差遣長子到世上來的時候,又說:“ 神所有的天使都要拜他。” \\ \\
Of the angels he says, "Who makes his angels winds, and his servants flames of fire." \\
 論到天使,說:“ 神以他的天使為風,以他的僕役為火燄。” \\ \\
But of the Son he says, "Thy throne, O God, is for ever and ever, the righteous scepter is the scepter of thy kingdom. \\
 但是論到兒子,卻說:“ 神啊!你的寶座是永永遠遠的,你國的權杖,是公平的權杖。 \\ \\
Thou hast loved righteousness and hated lawlessness; therefore God, thy God, has anointed thee with the oil of gladness beyond thy comrades." \\
 你喜愛公義,恨惡不法,所以, 神,就是你的 神,用喜樂的油膏抹你,勝過膏抹你的同伴。” \\ \\
And, "Thou, Lord, didst found the earth in the beginning, and the heavens are the work of thy hands; \\
 又說:“主啊!你起初立了地的根基,天也是你手的工作。 \\ \\
they will perish, but thou remainest; they will all grow old like a garment, \\
 天地都要毀滅,你卻長存;天地都要像衣服一樣漸漸殘舊, \\ \\
like a mantle thou wilt roll them up, and they will be changed. But thou art the same, and thy years will never end." \\
 你要把天地像外套一樣捲起來,天地就像衣服一樣被更換;只有你永不改變,你的年數也沒有窮盡。” \\ \\
But to what angel has he ever said, "Sit at my right hand, till I make thy enemies a stool for thy feet"? \\
  神可曾向哪一個天使說:“你坐在我的右邊,等我使你的仇敵作你的腳凳”呢? \\ \\
Are they not all ministering spirits sent forth to serve, for the sake of those who are to obtain salvation? \\
 天使不都是服役的靈,奉差遣為那些要承受救恩的人效勞嗎? \\ \\
\hline
\end{tabularx}

\newpage

\begin{tabularx}{\textwidth}{p{0.95\textwidth}}
\hline
Hebrews Ch. 2 \\
\hline \\
Therefore we must pay the closer attention to what we have heard, lest we drift away from it. \\
因此,我們必須更加密切注意所聽過的道理,免得我們隨流失去。 \\ \\
For if the message declared by angels was valid and every transgression or disobedience received a just retribution, \\
那透過天使所傳講的信息既然是確定的,所有干犯和不聽從的,都受了應得的報應。 \\ \\
how shall we escape if we neglect such a great salvation? It was declared at first by the Lord, and it was attested to us by those who heard him, \\
如果我們忽略了這麼大的救恩,怎麼能逃罪呢?這救恩起初是由主親自宣講的,後來聽見的人給我們證實了。 \\ \\
while God also bore witness by signs and wonders and various miracles and by gifts of the Holy Spirit distributed according to his own will. \\
 神又照著自己的旨意,用神蹟、奇事和各樣異能,以及聖靈的恩賜,與他們一同作見證。 \\ \\
For it was not to angels that God subjected the world to come, of which we are speaking. \\
 神並沒有把我們所說的“將來的世界”,交給天使管轄; \\ \\
It has been testified somewhere, "What is man that thou art mindful of him, or the son of man, that thou carest for him? \\
但是有人在聖經上某一處證實說:“人算甚麼,你竟記念他?世人算甚麼,你竟眷顧他? \\ \\
Thou didst make him for a little while lower than the angels, thou hast crowned him with glory and honor, \\
你使他暫時成了比天使卑微(“暫時成了比天使卑微”或譯“比天使稍低微一點”),卻賜給他榮耀尊貴作冠冕,(有些古卷在此有“並立他統管你手所造的一切”一句) \\ \\
putting everything in subjection under his feet." Now in putting everything in subjection to him, he left nothing outside his control. As it is, we do not yet see everything in subjection to him. \\
使萬物都服在他的腳下。”既然萬有都服了他,就沒有剩下一樣不服他的了。但是現在我們還沒有看見萬有都服他。 \\ \\
But we see Jesus, who for a little while was made lower than the angels, crowned with glory and honor because of the suffering of death, so that by the grace of God he might taste death for every one. \\
不過,我們看見那位暫時成了比天使卑微(“暫時成了比天使卑微”或譯“比天使稍低微一點”的耶穌,因為受了死的痛苦,就得了榮耀尊貴作冠冕,好叫他因著 神的恩典,為萬人嘗了死味。 \\ \\
For it was fitting that he, for whom and by whom all things exist, in bringing many sons to glory, should make the pioneer of their salvation perfect through suffering. \\
萬有因他而有、藉他而造的那位,為了要帶領許多兒子進入榮耀裡去,使救他們的元首藉著受苦而得到成全,本是合適的。 \\ \\
For he who sanctifies and those who are sanctified have all one origin. That is why he is not ashamed to call them brethren, \\
因為那位使人成聖的,和那些得到成聖的,同是出於一個源頭;所以他稱他們為弟兄也不以為恥。 \\ \\
saying, "I will proclaim thy name to my brethren, in the midst of the congregation I will praise thee." \\
他說:“我要向我的弟兄宣揚你的名,我要在聚會中歌頌你。” \\ \\
And again, "I will put my trust in him." And again, "Here am I, and the children God has given me." \\
又說:“我要信靠他。”又說:“看哪,我和 神所賜給我的孩子們。” \\ \\
Since therefore the children share in flesh and blood, he himself likewise partook of the same nature, that through death he might destroy him who has the power of death, that is, the devil, \\
孩子們既然同有血肉之體,他自己也照樣成為血肉之體,為要藉著死,消滅那掌握死權的魔鬼, \\ \\
and deliver all those who through fear of death were subject to lifelong bondage. \\
並且要釋放那些因為怕死而終身作奴僕的人。 \\ \\
For surely it is not with angels that he is concerned but with the descendants of Abraham. \\
其實,他並沒有救援天使,只救援亞伯拉罕的後裔。 \\ \\
Therefore he had to be made like his brethren in every respect, so that he might become a merciful and faithful high priest in the service of God, to make expiation for the sins of the people. \\
所以,他必須在各方面和他的弟兄們相同,為了要在 神的事上,成為仁慈忠信的大祭司,好為人民贖罪。 \\ \\
For because he himself has suffered and been tempted, he is able to help those who are tempted. \\
因為他自己既然經過試探,受了苦,就能夠幫助那些被試探的人。 \\ \\
\hline
\end{tabularx}

\newpage

\begin{tabularx}{\textwidth}{p{0.95\textwidth}}
\hline
Hebrews Ch. 3 \\
\hline \\
Therefore, holy brethren, who share in a heavenly call, consider Jesus, the apostle and high priest of our confession. \\
因此,同蒙天召的聖潔的弟兄啊!你們應該想想耶穌,就是作使徒、作我們所宣認的大祭司的那一位。 \\ \\
He was faithful to him who appointed him, just as Moses also was faithful in God's house. \\
他忠於那位委派他的,好像摩西在 神的全家盡忠一樣。 \\ \\
Yet Jesus has been counted worthy of as much more glory than Moses as the builder of a house has more honor than the house. \\
他比摩西配得更大的榮耀,好像建造房屋的人比房屋配得更大的尊貴一樣。 \\ \\
(For every house is built by some one, but the builder of all things is God.) \\
因為每一間房屋都是人建造的,只有萬物是 神建造的。 \\ \\
Now Moses was faithful in all God's house as a servant, to testify to the things that were to be spoken later, \\
摩西不過是個僕人,在 神的全家盡忠,為以後要傳講的事作證; \\ \\
but Christ was faithful over God's house as a son. And we are his house if we hold fast our confidence and pride in our hope. \\
但是基督卻是兒子,管理自己的家。如果我們把坦然無懼的心和可誇的盼望持守到底,我們就是他的家了。 \\ \\
Therefore, as the Holy Spirit says, "Today, when you hear his voice, \\
所以,就像聖靈所說的:“如果你們今天聽從他的聲音, \\ \\
do not harden your hearts as in the rebellion, on the day of testing in the wilderness, \\
就不要硬著心,好像在曠野惹他發怒、試探他的日子一樣; \\ \\
where your fathers put me to the test and saw my works for forty years. \\
在那裡,你們的祖先以試驗來試探我,觀看我的作為有四十年之久。 \\ \\
Therefore I was provoked with that generation, and said, `They always go astray in their hearts; they have not known my ways.' \\
所以,我向那個世代的人發怒,說:‘他們心裡常常迷誤,不認識我的道路。’ \\ \\
As I swore in my wrath, `They shall never enter my rest.'" \\
我就在烈怒中起誓,說:‘他們絕不可進入我的安息。’” \\ \\
Take care, brethren, lest there be in any of you an evil, unbelieving heart, leading you to fall away from the living God. \\
弟兄們,你們要小心,免得你們中間有人存著邪惡、不信的心,以致離棄了永活的 神; \\ \\
But exhort one another every day, as long as it is called "today," that none of you may be hardened by the deceitfulness of sin. \\
趁著還有叫作“今天”的時候,總要天天互相勸勉,免得你們中間有人受了罪惡的誘惑,心裡就剛硬了。 \\ \\
For we share in Christ, if only we hold our first confidence firm to the end, \\
如果我們把起初的信念堅持到底,就是有分於基督的人了。 \\ \\
while it is said, "Today, when you hear his voice, do not harden your hearts as in the rebellion." \\
經上說:“如果你們今天聽從他的聲音,就不要硬著心,像惹他發怒的時候一樣。” \\ \\
Who were they that heard and yet were rebellious? Was it not all those who left Egypt under the leadership of Moses? \\
那時,聽了他的話而惹他發怒的是誰呢?不就是摩西從埃及領出來的那些人嗎? \\ \\
And with whom was he provoked forty years? Was it not with those who sinned, whose bodies fell in the wilderness? \\
四十年之久, 神向誰發怒呢?不就是向那些犯了罪陳屍曠野的人嗎? \\ \\
And to whom did he swear that they should never enter his rest, but to those who were disobedient? \\
他又向誰起誓說,他們絕對不可以進入他的安息呢?不就是向那些不順從的人嗎? \\ \\
So we see that they were unable to enter because of unbelief. \\
這樣看來,他們不能進入安息,是因為不信的緣故。 \\ \\

\hline
\end{tabularx}

\newpage

\begin{tabularx}{\textwidth}{p{0.95\textwidth}}
\hline
Hebrews Ch. 4 \\
\hline \\
Therefore, while the promise of entering his rest remains, let us fear lest any of you be judged to have failed to reach it. \\
所以,那進入安息的應許,既然還給我們留著,我們就應該戰戰兢兢,恐怕我們中間有人像是被淘汰了。 \\ \\
For good news came to us just as to them; but the message which they heard did not benefit them, because it did not meet with faith in the hearers. \\
因為有福音傳給我們,像傳給他們一樣,只是他們所聽見的道,對他們沒有益處,因為他們沒有用信心與所聽見的打成一片(“沒有用信心與所聽見的打成一片”,有古卷作“沒有用信心與聽從這道的人打成一片”)。 \\ \\
For we who have believed enter that rest, as he has said, "As I swore in my wrath, `They shall never enter my rest,'" although his works were finished from the foundation of the world. \\
然而我們信了的人,就可以進入那安息。正如 神所說:“我在烈怒中起誓說,他們絕不可進入我的安息!”其實 神的工作,從創立世界以來已經完成了。 \\ \\
For he has somewhere spoken of the seventh day in this way, "And God rested on the seventh day from all his works." \\
因為論到第七日,他在聖經某一處說:“在第七日 神歇了他的一切工作。” \\ \\
And again in this place he said, "They shall never enter my rest." \\
但在這裡又說:“他們絕不可進入我的安息。” \\ \\
Since therefore it remains for some to enter it, and those who formerly received the good news failed to enter because of disobedience, \\
既然這安息還留著要讓一些人進去,但那些以前聽過福音的人,因為不順從不得進去; \\ \\
again he sets a certain day, "Today," saying through David so long afterward, in the words already quoted, "Today, when you hear his voice, do not harden your hearts." \\
所以 神就再定一個日子,就是過了很久以後,藉著大衛所說的“今天”,就像前面引用過的:“如果你們今天聽從他的聲音,就不要硬著心。” \\ \\
For if Joshua had given them rest, God would not speak later of another day. \\
如果約書亞已經使他們享受了安息, 神後來就不會再提到別的日子了。 \\ \\
So then, there remains a sabbath rest for the people of God; \\
這樣看來,為了 神的子民,必定另外有一個“安息日”的安息保留下來。 \\ \\
for whoever enters God's rest also ceases from his labors as God did from his. \\
因為那進入 神安息的人,就歇了自己的工作,好像 神歇了自己的工作一樣。 \\ \\
Let us therefore strive to enter that rest, that no one fall by the same sort of disobedience. \\
所以,我們要竭力進入那安息,免得有人隨著那不順從的樣子就跌倒了。 \\ \\
For the word of God is living and active, sharper than any two-edged sword, piercing to the division of soul and spirit, of joints and marrow, and discerning the thoughts and intentions of the heart. \\
因為 神的道是活的,是有效的,比一切兩刃的劍更鋒利,甚至可以刺入剖開魂與靈,關節與骨髓,並且能夠辨明心中的思想和意念。 \\ \\
And before him no creature is hidden, but all are open and laid bare to the eyes of him with whom we have to do. \\
被造的在 神面前沒有一樣不是顯明的,萬有在他的眼前都是赤露敞開的;我們必須向他交帳。 \\ \\
Since then we have a great high priest who has passed through the heavens, Jesus, the Son of God, let us hold fast our confession. \\
我們既然有一位偉大的、經過了眾天的大祭司,就是 神的兒子耶穌,就應該堅持所宣認的信仰。 \\ \\
For we have not a high priest who is unable to sympathize with our weaknesses, but one who in every respect has been tempted as we are, yet without sin. \\
因為我們的大祭司並不是不能同情我們的軟弱,他像我們一樣,也曾在各方面受過試探,只是他沒有犯罪。 \\ \\
Let us then with confidence draw near to the throne of grace, that we may receive mercy and find grace to help in time of need. \\
所以,我們只管坦然無懼地來到施恩的寶座前,為的是要領受憐憫,得到恩惠,作為及時的幫助。 \\ \\

\hline
\end{tabularx}

\newpage

\begin{tabularx}{\textwidth}{p{0.95\textwidth}}
\hline
Hebrews Ch. 5 \\
\hline \\
For every high priest chosen from among men is appointed to act on behalf of men in relation to God, to offer gifts and sacrifices for sins. \\
每一個大祭司都是從人間選出來,奉派替人辦理與 神有關的事,為的是要獻上禮物和贖罪的祭物。 \\ \\
He can deal gently with the ignorant and wayward, since he himself is beset with weakness. \\
他能夠溫和地對待那些無知和迷誤的人,因為他自己也被軟弱所困。 \\ \\
Because of this he is bound to offer sacrifice for his own sins as well as for those of the people. \\
因此,他怎樣為人民的罪獻祭,也應該怎樣為自己的罪獻祭。 \\ \\
And one does not take the honor upon himself, but he is called by God, just as Aaron was. \\
沒有人可以自己取得這大祭司的尊榮,只有像亞倫一樣,蒙 神選召的才可以。 \\ \\
So also Christ did not exalt himself to be made a high priest, but was appointed by him who said to him, "Thou art my Son, today I have begotten thee"; \\
照樣,基督也沒有自己爭取作大祭司的尊榮,而是曾經對他說:“你是我的兒子,我今日生了你”的 神榮耀了他; \\ \\
as he says also in another place, "Thou art a priest for ever, after the order of Melchiz'edek." \\
就像他在另一處說:“你是照著麥基洗德的體系,永遠作祭司的。” \\ \\
In the days of his flesh, Jesus offered up prayers and supplications, with loud cries and tears, to him who was able to save him from death, and he was heard for his godly fear. \\
基督在世的時候,曾經流淚大聲禱告懇求那位能救他脫離死亡的 神;因著他的敬虔,就蒙了應允。 \\ \\
Although he was a Son, he learned obedience through what he suffered; \\
他雖然是兒子,還是因著所受的苦難學會了順從。 \\ \\
and being made perfect he became the source of eternal salvation to all who obey him, \\
他既然順從到底(“他既然順從到底”或譯:“他既然達到完全”),就成了所有順從他的人得到永遠救恩的根源; \\ \\
being designated by God a high priest after the order of Melchiz'edek. \\
而且蒙 神照著麥基洗德的體系,稱他為大祭司。 \\ \\
About this we have much to say which is hard to explain, since you have become dull of hearing. \\
論到這些事,我們有很多話要說,可是很難解釋;因為你們已經遲鈍了,聽不進去。 \\ \\
For though by this time you ought to be teachers, you need some one to teach you again the first principles of God's word. You need milk, not solid food; \\
到這個時候,你們應該已經作老師了;可是你們還需要有人再把 神道理的初步教導你們。你們成了只能吃奶而不能吃乾糧的人! \\ \\
for every one who lives on milk is unskilled in the word of righteousness, for he is a child. \\
凡是吃奶的,還是個嬰孩,對公義的道理沒有經歷; \\ \\
But solid food is for the mature, for those who have their faculties trained by practice to distinguish good from evil. \\
只有長大成人的,才能吃乾糧,他們的官能因為操練純熟,就能分辨是非了。 \\ \\

\hline
\end{tabularx}

\newpage

\begin{tabularx}{\textwidth}{p{0.95\textwidth}}
\hline
Hebrews Ch. 6 \\
\hline \\
Therefore let us leave the elementary doctrine of Christ and go on to maturity, not laying again a foundation of repentance from dead works and of faith toward God, \\
所以,我們應當離開基督初步的道理,努力進到成熟的地步,不必在懊悔致死的行為,信靠 神,洗禮,按手禮,死人復活,和永遠審判的教訓上再立根基。 \\ \\
with instruction about ablutions, the laying on of hands, the resurrection of the dead, and eternal judgment. \\
所以,我們應當離開基督初步的道理,努力進到成熟的地步,不必在懊悔致死的行為,信靠 神,洗禮,按手禮,死人復活,和永遠審判的教訓上再立根基。 \\ \\
And this we will do if God permits. \\
只要 神允許,我們就這樣作。 \\ \\
For it is impossible to restore again to repentance those who have once been enlightened, who have tasted the heavenly gift, and have become partakers of the Holy Spirit, \\
因為那些曾經蒙了光照,嘗過屬天的恩賜的滋味,與聖靈有分, \\ \\
and have tasted the goodness of the word of God and the powers of the age to come, \\
並且嘗過 神美善的道和來世的權能的人, \\ \\
if they then commit apostasy, since they crucify the Son of God on their own account and hold him up to contempt. \\
如果偏離了正道,就不可能再使他們重新悔改了。因為他們親自把 神的兒子再釘在十字架上,公然羞辱他。 \\ \\
For land which has drunk the rain that often falls upon it, and brings forth vegetation useful to those for whose sake it is cultivated, receives a blessing from God. \\
這就像一塊地,吸收了常常下在它上面的雨水,如果長出對種植的人有用的菜蔬,就從 神那裡得福。 \\ \\
But if it bears thorns and thistles, it is worthless and near to being cursed; its end is to be burned. \\
但如果這塊地長出荊棘和蒺藜來,就被廢棄,近於咒詛,結局就是焚燒。 \\ \\
Though we speak thus, yet in your case, beloved, we feel sure of better things that belong to salvation. \\
不過,親愛的弟兄們,我們雖然這樣說,但對於你們,我們卻深信你們有更好的表現,結局就是得救。 \\ \\
For God is not so unjust as to overlook your work and the love which you showed for his sake in serving the saints, as you still do. \\
因為 神並不是不公義,以致忘記了你們的工作,和你們為他的名所表現的愛心,就是你們以前服事聖徒,現在還是服事他們。 \\ \\
And we desire each one of you to show the same earnestness in realizing the full assurance of hope until the end, \\
我們深願你們各人都表現同樣的熱誠,一直到底,使你們的盼望可以完全實現, \\ \\
so that you may not be sluggish, but imitators of those who through faith and patience inherit the promises. \\
並且不要懶惰,卻要效法那些憑著信心和忍耐承受應許的人。 \\ \\
For when God made a promise to Abraham, since he had no one greater by whom to swear, he swore by himself, \\
 神應許亞伯拉罕的時候,因為沒有比自己更大的可以指著起誓,他就指著自己起誓, \\ \\
saying, "Surely I will bless you and multiply you." \\
說:“我必定賜福給你,必定使你的後裔繁多。” \\ \\
And thus Abraham, having patiently endured, obtained the promise. \\
這樣,亞伯拉罕耐心等待,終於獲得了所應許的。 \\ \\
Men indeed swear by a greater than themselves, and in all their disputes an oath is final for confirmation. \\
因為人起誓都是指著比自己大的起誓。這誓言就了結了他們中間的一切糾紛,作為保證。 \\ \\
So when God desired to show more convincingly to the heirs of the promise the unchangeable character of his purpose, he interposed with an oath, \\
照樣, 神定意向那些承受應許的人,更清楚地表明他的旨意是不更改的,就用起誓作保證。 \\ \\
so that through two unchangeable things, in which it is impossible that God should prove false, we who have fled for refuge might have strong encouragement to seize the hope set before us. \\
這兩件事是不能更改的,因為 神是決不說謊的。因此,我們這些逃進避難所的人,就大得安慰,抓緊那擺在我們面前的盼望。 \\ \\
We have this as a sure and steadfast anchor of the soul, a hope that enters into the inner shrine behind the curtain, \\
我們有這盼望,就像靈魂的錨,又穩當又堅固,通過幔子直進到裡面。 \\ \\
where Jesus has gone as a forerunner on our behalf, having become a high priest for ever after the order of Melchiz'edek. \\
耶穌已經為我們作先鋒進入了幔子裡面;他是照著麥基洗德的體系,成了永遠的大祭司。 \\ \\

\hline
\end{tabularx}

\newpage

\begin{tabularx}{\textwidth}{p{0.95\textwidth}}
\hline
Hebrews Ch. 7 \\
\hline \\
For this Melchiz'edek, king of Salem, priest of the Most High God, met Abraham returning from the slaughter of the kings and blessed him; \\
這麥基洗德就是撒冷王,又是至高 神的祭司。亞伯拉罕殺敗眾王回來的時候,麥基洗德迎接他,並且給他祝福。 \\ \\
and to him Abraham apportioned a tenth part of everything. He is first, by translation of his name, king of righteousness, and then he is also king of Salem, that is, king of peace. \\
亞伯拉罕也把自己得來的一切,拿出十分之一來給他。麥基洗德這名字翻譯出來,頭一個意思就是“公義的王”;其次是“撒冷王”,就是“平安的王”的意思。 \\ \\
He is without father or mother or genealogy, and has neither beginning of days nor end of life, but resembling the Son of God he continues a priest for ever. \\
他沒有父親,沒有母親,沒有族譜,也沒有生死的記錄,而是與 神的兒子相似,永遠作祭司。 \\ \\
See how great he is! Abraham the patriarch gave him a tithe of the spoils. \\
你們想一想這人是多麼偉大啊!祖先亞伯拉罕也要從上等的擄物中,拿出十分之一來給了他。 \\ \\
And those descendants of Levi who receive the priestly office have a commandment in the law to take tithes from the people, that is, from their brethren, though these also are descended from Abraham. \\
那些領受祭司職分的利未子孫,奉命按照律法向人民,就是自己的弟兄,收取十分之一;雖然他們都是出於亞伯拉罕的。 \\ \\
But this man who has not their genealogy received tithes from Abraham and blessed him who had the promises. \\
可是那不與他們同譜系的麥基洗德,反而收納了亞伯拉罕的十分之一,並且給這蒙受應許的人祝福。 \\ \\
It is beyond dispute that the inferior is blessed by the superior. \\
向來都是位分大的給位分小的祝福,這是毫無疑問的。 \\ \\
Here tithes are received by mortal men; there, by one of whom it is testified that he lives. \\
在這裡,收取十分之一的,都是必死的;但在那裡,收納十分之一的,卻被證實是一位活著的。 \\ \\
One might even say that Levi himself, who receives tithes, paid tithes through Abraham, \\
並且可以這樣說,連那收取十分之一的利未,也透過亞伯拉罕繳納了十分之一。 \\ \\
for he was still in the loins of his ancestor when Melchiz'edek met him. \\
因為麥基洗德迎接亞伯拉罕的時候,利未還在他祖先的身體裡面。 \\ \\
Now if perfection had been attainable through the Levit'ical priesthood (for under it the people received the law), what further need would there have been for another priest to arise after the order of Melchiz'edek, rather than one named after the order of Aaron? \\
這樣看來,如果藉著利未人的祭司制度能達到完全的地步(人民是在這制度下領受律法的),為甚麼還需要照著麥基洗德的體系,另外興起一位祭司,而不照著亞倫的體系呢? \\ \\
For when there is a change in the priesthood, there is necessarily a change in the law as well. \\
祭司的制度既然更改了,律法也必須更改。 \\ \\
For the one of whom these things are spoken belonged to another tribe, from which no one has ever served at the altar. \\
因為這些話所指的那位,原是屬於另外一個支派的,這支派向來沒有人在祭壇前供職。 \\ \\
For it is evident that our Lord was descended from Judah, and in connection with that tribe Moses said nothing about priests. \\
我們的主明明是從猶大支派出來的,關於這個支派,摩西並沒有提及祭司的事。 \\ \\
This becomes even more evident when another priest arises in the likeness of Melchiz'edek, \\
如果有另一位像麥基洗德那樣的祭司興起來,那麼,這裡所說的就更明顯了。 \\ \\
who has become a priest, not according to a legal requirement concerning bodily descent but by the power of an indestructible life. \\
他成了祭司,不是按著律法上肉身的條例,卻是按著不能毀壞的生命的大能。 \\ \\
For it is witnessed of him, "Thou art a priest for ever, after the order of Melchiz'edek." \\
因為有為他作證的說:“你永遠作祭司,是照著麥基洗德的體系。” \\ \\
On the one hand, a former commandment is set aside because of its weakness and uselessness \\
一方面,從前的條例因為軟弱,沒有用處,就廢棄了; \\ \\
(for the law made nothing perfect); on the other hand, a better hope is introduced, through which we draw near to God. \\
(因為律法從來沒有使甚麼得到完全,)另一方面,它卻帶來了更美的盼望,藉著這盼望,我們就可以親近 神。 \\ \\

\hline
\end{tabularx}

\newpage
\begin{tabularx}{\textwidth}{p{0.95\textwidth}}
\hline
Hebrews Ch. 7 continue \\
\hline \\
And it was not without an oath. \\
此外,還有關於誓言的事。其他成為祭司的,並不是用誓言立的; \\ \\
Those who formerly became priests took their office without an oath, but this one was addressed with an oath, "The Lord has sworn and will not change his mind, `Thou art a priest for ever.'" \\
只有耶穌是用誓言立的,因為那立他的對他說:“主已經起了誓,決不改變,你永遠作祭司。” \\ \\
This makes Jesus the surety of a better covenant. \\
耶穌既然是用誓言立的,就成了更美好的約的保證。 \\ \\
The former priests were many in number, because they were prevented by death from continuing in office; \\
一方面,從前那些作祭司的,因為受死亡的限制,不能長久留任,所以人數眾多。 \\ \\
but he holds his priesthood permanently, because he continues for ever. \\
另一方面,因為耶穌是永遠長存的,就擁有他永不更改的祭司職位。 \\ \\
Consequently he is able for all time to save those who draw near to God through him, since he always lives to make intercession for them. \\
因此,那些靠著他進到 神面前的人,他都能拯救到底;因為他長遠活著,為他們代求。 \\ \\
For it was fitting that we should have such a high priest, holy, blameless, unstained, separated from sinners, exalted above the heavens. \\
這樣的一位大祭司,對我們本是合適的。他是聖潔、沒有邪惡、沒有玷污、從罪人中分別出來、高過眾天的。 \\ \\
He has no need, like those high priests, to offer sacrifices daily, first for his own sins and then for those of the people; he did this once for all when he offered up himself. \\
他不必像那些大祭司,天天先為自己的罪獻祭,然後為人民的罪獻祭;因為他獻上了自己,就把這事一次而永遠的成全了。 \\ \\
Indeed, the law appoints men in their weakness as high priests, but the word of the oath, which came later than the law, appoints a Son who has been made perfect for ever. \\
律法所立的大祭司,都是軟弱的人;可是在律法以後,用誓言所立的兒子,卻是成為完全直到永遠的。 \\ \\

\hline
\end{tabularx}

\newpage
\begin{tabularx}{\textwidth}{p{0.95\textwidth}}
\hline
Hebrews Ch. 8 \\
\hline \\
Now the point in what we are saying is this: we have such a high priest, one who is seated at the right hand of the throne of the Majesty in heaven, \\
我們所講論的重點,就是我們有這樣的一位大祭司,他已經坐在眾天之上至尊者的寶座右邊, \\ \\
a minister in the sanctuary and the true tent which is set up not by man but by the Lord. \\
在至聖所和真會幕裡供職;這真會幕是主支搭的,不是人支搭的。 \\ \\
For every high priest is appointed to offer gifts and sacrifices; hence it is necessary for this priest also to have something to offer. \\
所有大祭司都是為了獻禮物和祭品而設立的,所以這位大祭司,也必須有所獻上的。 \\ \\
Now if he were on earth, he would not be a priest at all, since there are priests who offer gifts according to the law. \\
如果他在地上,就不會作祭司,因為已經有按照律法獻禮物的祭司了。 \\ \\
They serve a copy and shadow of the heavenly sanctuary; for when Moses was about to erect the tent, he was instructed by God, saying, "See that you make everything according to the pattern which was shown you on the mountain." \\
這些祭司所供奉的職事,不過是天上的事物的副本和影像,就如摩西將要造會幕的時候, 神曾經警告他說:“你要留心,各樣物件,都要照著在山上指示你的樣式去作。” \\ \\
But as it is, Christ has obtained a ministry which is as much more excellent than the old as the covenant he mediates is better, since it is enacted on better promises. \\
但是現在耶穌得了更尊貴的職分,正好像他是更美的約的中保,這約是憑著更美的應許立的。 \\ \\
For if that first covenant had been faultless, there would have been no occasion for a second. \\
如果頭一個約沒有缺點,就沒有尋求另一個約的必要了。 \\ \\
For he finds fault with them when he says:  "The days will come, says the Lord, when I will establish a new covenant with the house of Israel and with the house of Judah; \\
可是 神指責他們,說:“看哪,主說,日子要到了,我要與以色列家和猶大家訂立新約。 \\ \\
not like the covenant that I made with their fathers on the day when I took them by the hand to lead them out of the land of Egypt; for they did not continue in my covenant, and so I paid no heed to them, says the Lord. \\
這新約不像從前我拉他們祖先的手,領他們出埃及的日子與他們所立的約。因為他們沒有遵守我的約,我就不理會他們。這是主說的。 \\ \\
This is the covenant that I will make with the house of Israel after those days, says the Lord: I will put my laws into their minds, and write them on their hearts, and I will be their God, and they shall be my people. \\
主說:‘因為在那些日子以後,我要與以色列家所立的約是這樣:我要把我的律法放在他們的心思裡面,寫在他們的心上。我要作他們的 神,他們要作我的子民。 \\ \\
And they shall not teach every one his fellow or every one his brother, saying, `Know the Lord,' for all shall know me, from the least of them to the greatest. \\
他們各人必不用教導自己的鄰居,和自己的同胞,說:你要認識主。因為所有的人,從最小到最大的,都必認識我。 \\ \\
For I will be merciful toward their iniquities, and I will remember their sins no more." \\
我也要寬恕他們的不義,決不再記著他們的罪惡。’” \\ \\
In speaking of a new covenant he treats the first as obsolete. And what is becoming obsolete and growing old is ready to vanish away. \\
 神既然說到新的約,就是把前約當作舊的了;那變成陳舊衰老的,就快要消逝了。 \\ \\

\hline
\end{tabularx}

\newpage
\begin{tabularx}{\textwidth}{p{0.95\textwidth}}
\hline
Hebrews Ch. 9 \\
\hline \\
Now even the first covenant had regulations for worship and an earthly sanctuary. \\
前約也有它敬拜的規例,和屬世界的聖所。 \\ \\
For a tent was prepared, the outer one, in which were the lampstand and the table and the bread of the Presence; it is called the Holy Place. \\
因為有一個支搭好了的會幕,第一進叫作聖所,裡面有燈臺、桌子和陳設餅。 \\ \\
Behind the second curtain stood a tent called the Holy of Holies, \\
在第二層幔子後面還有一個會幕,叫作至聖所, \\ \\
having the golden altar of incense and the ark of the covenant covered on all sides with gold, which contained a golden urn holding the manna, and Aaron's rod that budded, and the tables of the covenant; \\
裡面有金香壇,有全部包金的約櫃,櫃裡有盛著嗎哪的金罐、亞倫那發過芽的杖和兩塊約板。 \\ \\
above it were the cherubim of glory overshadowing the mercy seat. Of these things we cannot now speak in detail. \\
櫃的上面有榮耀的基路伯罩著施恩座,關於這一切,現在不能一一細說了。 \\ \\
These preparations having thus been made, the priests go continually into the outer tent, performing their ritual duties; \\
這一切物件都這樣預備好了,祭司就常常進入第一進會幕,執行敬拜的事。 \\ \\
but into the second only the high priest goes, and he but once a year, and not without taking blood which he offers for himself and for the errors of the people. \\
至於第二進會幕,只有大祭司一年一次獨自進去,並且非帶著血不可,好為自己和人民的愚妄把血獻上。 \\ \\
By this the Holy Spirit indicates that the way into the sanctuary is not yet opened as long as the outer tent is still standing \\
聖靈藉著這事表明,當第一進會幕存在的時候,進入至聖所的路,還沒有顯明出來。 \\ \\
(which is symbolic for the present age). According to this arrangement, gifts and sacrifices are offered which cannot perfect the conscience of the worshiper, \\
這第一進會幕是現今的時代的預表,其實所獻的禮物和祭品,都不能使敬拜的人在良心上得到完全。 \\ \\
but deal only with food and drink and various ablutions, regulations for the body imposed until the time of reformation. \\
這些只是關於飲食和各樣潔淨的禮儀,是在“更新的時候”來到之前,為肉體立的規例。 \\ \\
But when Christ appeared as a high priest of the good things that have come, then through the greater and more perfect tent (not made with hands, that is, not of this creation) \\
但基督已經來了,作了已經實現的美好事物的大祭司;他經過更大、更完備的會幕(不是人手所做的,也就是不屬於這被造的世界的)。 \\ \\
he entered once for all into the Holy Place, taking not the blood of goats and calves but his own blood, thus securing an eternal redemption. \\
他不是用山羊和牛犢的血,而是用自己的血,只一次進了至聖所,就得到了永遠的救贖。 \\ \\
For if the sprinkling of defiled persons with the blood of goats and bulls and with the ashes of a heifer sanctifies for the purification of the flesh, \\
如果山羊和公牛的血,以及母牛犢的灰,灑在不潔的人身上,尚且可以使他們成為聖潔,身體潔淨, \\ \\
how much more shall the blood of Christ, who through the eternal Spirit offered himself without blemish to God, purify your conscience from dead works to serve the living God. \\
何況基督的血呢?他藉著永遠的靈,把自己無瑕無疵的獻給 神,他的血不是更能潔淨我們的良心脫離致死的行為,使我們可以事奉永活的 神嗎? \\ \\
Therefore he is the mediator of a new covenant, so that those who are called may receive the promised eternal inheritance, since a death has occurred which redeems them from the transgressions under the first covenant. \\
因此,他作了新約的中保,藉著他的死,使人在前約之下的過犯得到救贖,就叫那些蒙召的人,得著永遠基業的應許。 \\ \\
For where a will is involved, the death of the one who made it must be established. \\
凡有遺囑(“遺囑”或譯:“約”,與17節同),必須證實立遺囑的人死了; \\ \\
For a will takes effect only at death, since it is not in force as long as the one who made it is alive. \\
因為人死了,遺囑才能確立,立遺囑的人還活著的時候,遺囑決不生效。 \\ \\
Hence even the first covenant was not ratified without blood. \\
因此,前約並不是沒有用血立的: \\ \\

\hline
\end{tabularx}

\newpage
\begin{tabularx}{\textwidth}{p{0.95\textwidth}}
\hline
Hebrews Ch. 9 continue \\
\hline \\
For when every commandment of the law had been declared by Moses to all the people, he took the blood of calves and goats, with water and scarlet wool and hyssop, and sprinkled both the book itself and all the people, \\
當日摩西按照律法,向所有人民宣布了各樣的誡命,就拿牛犢(好些抄本在此有“和山羊”)的血和水,用朱紅色的羊毛與牛膝草,灑在律法書上和人民身上, \\ \\
saying, "This is the blood of the covenant which God commanded you." \\
說:“這就是 神規定你們立約的血。” \\ \\
And in the same way he sprinkled with the blood both the tent and all the vessels used in worship. \\
他照樣把血灑在會幕和各樣應用的器皿上。 \\ \\
Indeed, under the law almost everything is purified with blood, and without the shedding of blood there is no forgiveness of sins. \\
按著律法,幾乎所有都是用血潔淨的,如果沒有流血,就沒有赦免。 \\ \\
Thus it was necessary for the copies of the heavenly things to be purified with these rites, but the heavenly things themselves with better sacrifices than these. \\
照著天上樣式作的既然必須這樣去潔淨,天上物體的本身,就應該用更美的祭品去潔淨了。 \\ \\
For Christ has entered, not into a sanctuary made with hands, a copy of the true one, but into heaven itself, now to appear in the presence of God on our behalf. \\
因為基督不是進了人手所做的聖所(那不過是真聖所的表象),而是進到天上,現在替我們顯露在 神的面前。 \\ \\
Nor was it to offer himself repeatedly, as the high priest enters the Holy Place yearly with blood not his own; \\
他不必多次把自己獻上,好像大祭司每年帶著不是自己的血進入至聖所一樣。 \\ \\
for then he would have had to suffer repeatedly since the foundation of the world. But as it is, he has appeared once for all at the end of the age to put away sin by the sacrifice of himself. \\
如果這樣,他從創世以來,就必須受許多次的苦了。可是現在他在這世代的終結,只顯現一次,把自己作為祭品獻上,好除掉罪。 \\ \\
And just as it is appointed for men to die once, and after that comes judgment, \\
按著定命,人人都要死一次,死後還有審判。 \\ \\
so Christ, having been offered once to bear the sins of many, will appear a second time, not to deal with sin but to save those who are eagerly waiting for him. \\
照樣,基督為了擔當許多人的罪,也曾經一次把自己獻上;將來他還要再一次顯現,不是為擔當罪,而是要向那些熱切期待他的人成全救恩。 \\ \\

\hline
\end{tabularx}

\newpage

\begin{tabularx}{\textwidth}{p{0.95\textwidth}}
\hline
Hebrews Ch. 10 \\
\hline \\
For since the law has but a shadow of the good things to come instead of the true form of these realities, it can never, by the same sacrifices which are continually offered year after year, make perfect those who draw near. \\
律法既然是以後要來的美好事物的影子,不是本體的真象,就不能憑著每年獻同樣的祭品,使那些進前來的人得到完全。 \\ \\
Otherwise, would they not have ceased to be offered? If the worshipers had once been cleansed, they would no longer have any consciousness of sin. \\
如果敬拜的人一次得潔淨,良心就不再覺得有罪,那麼,獻祭的事不是早就停止了嗎? \\ \\
But in these sacrifices there is a reminder of sin year after year. \\
可是那些祭品,卻使人每年都想起罪來, \\ \\
For it is impossible that the blood of bulls and goats should take away sins. \\
因為公牛和山羊的血不能把罪除去。 \\ \\
Consequently, when Christ came into the world, he said,  "Sacrifices and offerings thou hast not desired, but a body hast thou prepared for me; \\
所以,基督到世上來的時候,就說:“祭品和禮物不是你所要的,你卻為我預備了身體。 \\ \\
in burnt offerings and sin offerings thou hast taken no pleasure. \\
燔祭和贖罪祭,不是你所喜悅的; \\ \\
Then I said, `Lo, I have come to do thy will, O God,' as it is written of me in the roll of the book." \\
那時我說:‘看哪!我來了,經卷上已經記載我的事, 神啊!我來是要遵行你的旨意。’” \\ \\
When he said above, "Thou hast neither desired nor taken pleasure in sacrifices and offerings and burnt offerings and sin offerings" (these are offered according to the law), \\
前面說:“祭品和禮物,燔祭和贖罪祭,不是你所要的,也不是你所喜悅的。”這些都是按照律法獻的; \\ \\
then he added, "Lo, I have come to do thy will." He abolishes the first in order to establish the second. \\
接著又說:“看哪!我來了,是要遵行你的旨意。”可見他廢除那先前的,為要建立那後來的。 \\ \\
And by that will we have been sanctified through the offering of the body of Jesus Christ once for all. \\
我們憑著這旨意,藉著耶穌基督一次獻上他的身體,就已經成聖。 \\ \\
And every priest stands daily at his service, offering repeatedly the same sacrifices, which can never take away sins. \\
所有的祭司都是天天站著事奉,多次獻上同樣的祭品,那些祭品永遠不能把罪除去。 \\ \\
But when Christ had offered for all time a single sacrifice for sins, he sat down at the right hand of God, \\
唯有基督獻上了一次永遠有效的贖罪祭,就在 神的右邊坐下來。 \\ \\
then to wait until his enemies should be made a stool for his feet. \\
此後,只是等待 神把他的仇敵放在他的腳下,作他的腳凳。 \\ \\
For by a single offering he has perfected for all time those who are sanctified. \\
因為他獻上了一次的祭,就使那些成聖的人永遠得到完全。 \\ \\
And the Holy Spirit also bears witness to us; for after saying, \\
聖靈也向我們作見證,因為後來他說過: \\ \\
"This is the covenant that I will make with them after those days, says the Lord: I will put my laws on their hearts, and write them on their minds," \\
“主說:‘在那些日子以後,我要與他們所立的約是這樣:我要把我的律法放在他們的心思裡面,寫在他們的心上。’” \\ \\
then he adds, "I will remember their sins and their misdeeds no more." \\
又說:“我決不再記著他們的罪惡,和不法的行為。” \\ \\
Where there is forgiveness of these, there is no longer any offering for sin. \\
這一切既然都赦免了,就不必再為罪獻祭了。 \\ \\

\hline
\end{tabularx}

\newpage

\begin{tabularx}{\textwidth}{p{0.95\textwidth}}
\hline
Hebrews Ch. 10 continue \\
\hline \\
Therefore, brethren, since we have confidence to enter the sanctuary by the blood of Jesus, \\
所以,弟兄們!我們憑著耶穌的血,可以坦然無懼地進入至聖所。 \\ \\
by the new and living way which he opened for us through the curtain, that is, through his flesh, \\
這進入的路,是他給我們開闢的,是一條通過幔子、又新又活的路,這幔子就是他的身體。 \\ \\
and since we have a great priest over the house of God, \\
我們既然有一位偉大的祭司治理 神的家, \\ \\
let us draw near with a true heart in full assurance of faith, with our hearts sprinkled clean from an evil conscience and our bodies washed with pure water. \\
我們良心的邪惡既然被灑淨,身體也用清水洗淨了,那麼,我們就應該懷著真誠的心和完備的信,進到 神面前; \\ \\
Let us hold fast the confession of our hope without wavering, for he who promised is faithful; \\
又應該堅持我們所宣認的盼望,毫不動搖,因為那應許我們的是信實的。 \\ \\
and let us consider how to stir up one another to love and good works, \\
我們又應該彼此關心,激發愛心,勉勵行善。 \\ \\
not neglecting to meet together, as is the habit of some, but encouraging one another, and all the more as you see the Day drawing near. \\
我們不可放棄聚會,好像有些人的習慣一樣;卻要互相勸勉。你們既然知道那日子臨近,就更應該這樣。 \\ \\
For if we sin deliberately after receiving the knowledge of the truth, there no longer remains a sacrifice for sins, \\
如果我們領受了真理的知識以後,還是故意犯罪,就再沒有留下贖罪的祭品了; \\ \\
but a fearful prospect of judgment, and a fury of fire which will consume the adversaries. \\
只好恐懼地等待著審判,和那快要吞滅眾仇敵的烈火。 \\ \\
A man who has violated the law of Moses dies without mercy at the testimony of two or three witnesses. \\
如果有人干犯了摩西的律法,憑著兩三個證人,他尚且得不到憐憫而死; \\ \\
How much worse punishment do you think will be deserved by the man who has spurned the Son of God, and profaned the blood of the covenant by which he was sanctified, and outraged the Spirit of grace? \\
何況是踐踏 神的兒子,把那使他成聖的立約的血當作俗物,又侮辱施恩的聖靈的人,你們想想,他不是應該受更嚴厲的刑罰嗎? \\ \\
For we know him who said, "Vengeance is mine, I will repay." And again, "The Lord will judge his people." \\
因為我們知道誰說過:“伸冤在我,我必報應。”又說:“主必定審判他自己的子民。” \\ \\
It is a fearful thing to fall into the hands of the living God. \\
落在永活的 神手裡,真是可怕的。 \\ \\
But recall the former days when, after you were enlightened, you endured a hard struggle with sufferings, \\
你們要回想從前的日子,那時,你們蒙了光照,忍受了許多痛苦的煎熬; \\ \\
sometimes being publicly exposed to abuse and affliction, and sometimes being partners with those so treated. \\
有時在眾人面前被辱罵,遭患難;有時卻成了遭遇同樣情形的人的同伴。 \\ \\
For you had compassion on the prisoners, and you joyfully accepted the plundering of your property, since you knew that you yourselves had a better possession and an abiding one. \\
你們同情那些遭監禁的人;你們的家業被搶奪的時候,又以喜樂的心接受,因為知道自己有更美長存的家業。 \\ \\
Therefore do not throw away your confidence, which has a great reward. \\
所以,你們不可丟棄坦然無懼的心,這樣的心是帶有大賞賜的。 \\ \\
For you have need of endurance, so that you may do the will of God and receive what is promised. \\
你們還需要忍耐,好使你們行完了 神的旨意,可以領受所應許的。 \\ \\
"For yet a little while, and the coming one shall come and shall not tarry; \\
因為:“還有一點點的時候,那要來的就來,並不遲延。 \\ \\
but my righteous one shall live by faith, and if he shrinks back, my soul has no pleasure in him." \\
我的義人必因信得生,如果他後退,我的心就不喜悅他。” \\ \\
But we are not of those who shrink back and are destroyed, but of those who have faith and keep their souls. \\
但我們不是那些後退以致滅亡的人,而是有信心以致保全生命的人。 \\ \\

\hline
\end{tabularx}

\newpage

\begin{tabularx}{\textwidth}{p{0.95\textwidth}}
\hline
Hebrews Ch. 11 \\
\hline \\
Now faith is the assurance of things hoped for, the conviction of things not seen. \\
信就是對所盼望的事的把握,是看不見之事的明證。 \\ \\
For by it the men of old received divine approval. \\
因著這信心,古人得到了稱許。 \\ \\
By faith we understand that the world was created by the word of God, so that what is seen was made out of things which do not appear. \\
因著信,我們就明白宇宙(“宇宙”或譯:“諸世界”或“眾世代”)是因著 神的話造成的。這樣,那看得見的就是從那看不見的造出來的。 \\ \\
By faith Abel offered to God a more acceptable sacrifice than Cain, through which he received approval as righteous, God bearing witness by accepting his gifts; he died, but through his faith he is still speaking. \\
因著信,亞伯比該隱獻上更美的祭品給 神;藉著這信心,他被 神稱許為義人,這是 神指著他的禮物所作的見證;他雖然死了,卻藉著信仍然說話。 \\ \\
By faith Enoch was taken up so that he should not see death; and he was not found, because God had taken him. Now before he was taken he was attested as having pleased God. \\
因著信,以諾被遷去了,使他不至於死,人也找不著他,因為 神把他遷去了。原來在遷去以前,他已經得了 神喜悅他的明證。 \\ \\
And without faith it is impossible to please him. For whoever would draw near to God must believe that he exists and that he rewards those who seek him. \\
沒有信,就不能得到 神的喜悅;因為來到 神面前的人,必須信 神存在,並且信他會賞賜那些尋求他的人。 \\ \\
By faith Noah, being warned by God concerning events as yet unseen, took heed and constructed an ark for the saving of his household; by this he condemned the world and became an heir of the righteousness which comes by faith. \\
因著信,挪亞在還沒有看見的事上,得了 神的警告,就動了敬畏的心,做了一艘方舟,使他全家得救。藉著這信心,他就定了那世代的罪,自己也承受了那因信而來的義。 \\ \\
By faith Abraham obeyed when he was called to go out to a place which he was to receive as an inheritance; and he went out, not knowing where he was to go. \\
因著信,亞伯拉罕在蒙召的時候,就聽命往他將要承受為業的地方去;他出去的時候,還不知道要往哪裡去。 \\ \\
By faith he sojourned in the land of promise, as in a foreign land, living in tents with Isaac and Jacob, heirs with him of the same promise. \\
因著信,他在應許之地寄居,好像是在異鄉,與承受同樣應許的以撒、雅各一樣住在帳棚裡。 \\ \\
For he looked forward to the city which has foundations, whose builder and maker is God. \\
因為他等待那座有根基的城,就是 神所設計所建造的。 \\ \\
By faith Sarah herself received power to conceive, even when she was past the age, since she considered him faithful who had promised. \\
因著信,甚至撒拉,她雖然過了生育的年齡,還是能夠懷孕,因為她認為那應許她的是信實的。 \\ \\
Therefore from one man, and him as good as dead, were born descendants as many as the stars of heaven and as the innumerable grains of sand by the seashore. \\
所以從一個好像已死的人,竟然生出許多子孫來,仿佛天上的星那麼眾多,海邊的沙那麼無數。 \\ \\
These all died in faith, not having received what was promised, but having seen it and greeted it from afar, and having acknowledged that they were strangers and exiles on the earth. \\
這些人都是存著信心死了的,還沒有得著所應許的,只不過是從遠處看見,就表示歡迎,又承認他們在世上是異鄉人,是客旅。 \\ \\
For people who speak thus make it clear that they are seeking a homeland. \\
因為說這樣話的人,是表明他們在尋求一個家鄉。 \\ \\
If they had been thinking of that land from which they had gone out, they would have had opportunity to return. \\
如果他們懷念已經離開了的地方,還有可以回去的機會。 \\ \\
But as it is, they desire a better country, that is, a heavenly one. Therefore God is not ashamed to be called their God, for he has prepared for them a city. \\
但是現在他們所嚮往的,是一個更美的、在天上的家鄉。所以, 神不以他們稱他為 神而覺得羞恥;因為他已經為他們預備了一座城。 \\ \\

\hline
\end{tabularx}

\newpage

\begin{tabularx}{\textwidth}{p{0.95\textwidth}}
\hline
Hebrews Ch. 11 continue \\
\hline \\
By faith Abraham, when he was tested, offered up Isaac, and he who had received the promises was ready to offer up his only son, \\
因著信,亞伯拉罕在受試驗的時候,就把以撒獻上;這就是那歡喜領受應許的人,獻上了自己的獨生子; \\ \\
of whom it was said, "Through Isaac shall your descendants be named." \\
論到這個兒子,曾經有話說:“以撒生的,才可以稱為你的後裔。” \\ \\
He considered that God was able to raise men even from the dead; hence, figuratively speaking, he did receive him back. \\
亞伯拉罕認定, 神能使人從死人中復活,因此,就喻意說,他的確從死裡得回他的兒子。 \\ \\
By faith Isaac invoked future blessings on Jacob and Esau. \\
因著信,以撒給雅各和以掃祝福,論到將來的事。 \\ \\
By faith Jacob, when dying, blessed each of the sons of Joseph, bowing in worship over the head of his staff. \\
因著信,雅各臨死的時候,分別為約瑟的兒子祝福,又倚著杖頭敬拜 神。 \\ \\
By faith Joseph, at the end of his life, made mention of the exodus of the Israelites and gave directions concerning his burial. \\
因著信,約瑟臨終的時候,提到以色列子民出埃及的事,並且為自己的骸骨留下遺言。 \\ \\
By faith Moses, when he was born, was hid for three months by his parents, because they saw that the child was beautiful; and they were not afraid of the king's edict. \\
因著信,摩西的父母在摩西生下來以後,因為看見孩子俊美,就把他藏了三個月,不怕王的命令。 \\ \\
By faith Moses, when he was grown up, refused to be called the son of Pharaoh's daughter, \\
因著信,摩西長大了以後,就拒絕被稱為法老女兒的兒子。 \\ \\
choosing rather to share ill-treatment with the people of God than to enjoy the fleeting pleasures of sin. \\
他寧願選擇和 神的子民一同受苦,也不肯享受罪惡中暫時的快樂。 \\ \\
He considered abuse suffered for the Christ greater wealth than the treasures of Egypt, for he looked to the reward. \\
在他看來,為著基督受的凌辱,比埃及的財物更寶貴,因為他注視將來的賞賜。 \\ \\
By faith he left Egypt, not being afraid of the anger of the king; for he endured as seeing him who is invisible. \\
因著信,他離開了埃及,不怕王的忿怒;因為他堅定不移,就像看見了人不能看見的 神。 \\ \\
By faith he kept the Passover and sprinkled the blood, so that the Destroyer of the first-born might not touch them. \\
因著信,他立了逾越節和灑血的禮,免得那滅命的侵犯以色列人的長子。 \\ \\
By faith the people crossed the Red Sea as if on dry land; but the Egyptians, when they attempted to do the same, were drowned. \\
因著信,他們走過了紅海,好像走過旱地一樣;埃及人也試著要過去,就被淹沒了。 \\ \\
By faith the walls of Jericho fell down after they had been encircled for seven days. \\
因著信,耶利哥的城牆被圍繞了七天,就倒塌了。 \\ \\
By faith Rahab the harlot did not perish with those who were disobedient, because she had given friendly welcome to the spies. \\
因著信,妓女喇合和和平平接待了偵察的人,就沒有和那些不順從的人一起滅亡。 \\ \\
And what more shall I say? For time would fail me to tell of Gideon, Barak, Samson, Jephthah, of David and Samuel and the prophets -- \\
我還要再說甚麼呢?如果再要述說基甸、巴拉、參孫、耶弗他、大衛、撒母耳和眾先知的事,時間就不夠了。 \\ \\
who through faith conquered kingdoms, enforced justice, received promises, stopped the mouths of lions, \\
他們藉著信,就戰勝了敵國,伸張了正義,得到了應許,堵住了獅子的口, \\ \\
quenched raging fire, escaped the edge of the sword, won strength out of weakness, became mighty in war, put foreign armies to flight. \\
消滅了烈火的威力,逃脫了刀劍的鋒刃,軟弱變成剛強,在戰爭中顯出大能,把外國的軍隊擊退。 \\ \\
Women received their dead by resurrection. Some were tortured, refusing to accept release, that they might rise again to a better life. \\
有些婦女得回從死裡復活的親人;但也有些人忍受了酷刑,不肯接受釋放,為的是要得著更美的復活。 \\ \\
Others suffered mocking and scourging, and even chains and imprisonment. \\
又有些人遭受了戲弄、鞭打,甚至捆鎖、監禁; \\ \\
They were stoned, they were sawn in two, they were killed with the sword; they went about in skins of sheep and goats, destitute, afflicted, ill-treated -- \\
被石頭打死,被鋸鋸死,(後期抄本在此加上“受試探”)被刀殺死。他們披著綿羊山羊的皮到處奔跑、受窮乏、遭患難、被虐待; \\ \\
of whom the world was not worthy -- wandering over deserts and mountains, and in dens and caves of the earth. \\
原是這世界不配有的人。他們飄流無定,在曠野、山嶺、石洞和地穴棲身。 \\ \\
And all these, though well attested by their faith, did not receive what was promised, \\
所有這些人都藉著信得了稱許,卻還沒有得著所應許的; \\ \\
since God had foreseen something better for us, that apart from us they should not be made perfect. \\
因為 神已經為我們預備了更美的事,使他們若不跟我們在一起,就不能完全。 \\ \\

\hline
\end{tabularx}

\newpage

\begin{tabularx}{\textwidth}{p{0.95\textwidth}}
\hline
Hebrews Ch. 12 \\
\hline \\
Therefore, since we are surrounded by so great a cloud of witnesses, let us also lay aside every weight, and sin which clings so closely, and let us run with perseverance the race that is set before us, \\
所以,我們既然有這麼多的見證人,像雲彩圍繞著我們,就應該脫下各樣的拖累,和容易纏住我們的罪,以堅忍的心奔跑那擺在我們面前的賽程; \\ \\
looking to Jesus the pioneer and perfecter of our faith, who for the joy that was set before him endured the cross, despising the shame, and is seated at the right hand of the throne of God. \\
專一注視耶穌,就是那位信心的創始者和完成者。他因為那擺在面前的喜樂,就忍受了十字架,輕看了羞辱,現在就坐在 神寶座的右邊。 \\ \\
Consider him who endured from sinners such hostility against himself, so that you may not grow weary or fainthearted. \\
這位忍受罪人那樣頂撞的耶穌,你們要仔細思想,免得疲倦灰心。 \\ \\
In your struggle against sin you have not yet resisted to the point of shedding your blood. \\
你們與罪惡鬥爭,還沒有對抗到流血的地步; \\ \\
And have you forgotten the exhortation which addresses you as sons? -- "My son, do not regard lightly the discipline of the Lord, nor lose courage when you are punished by him. \\
你們又忘記了那勸你們好像勸兒子的話,說:“我兒!你不可輕看主的管教,受責備的時候也不要灰心; \\ \\
For the Lord disciplines him whom he loves, and chastises every son whom he receives." \\
因為主所愛的,他必管教,他又鞭打所收納的每一個兒子。” \\ \\
It is for discipline that you have to endure. God is treating you as sons; for what son is there whom his father does not discipline? \\
為了接受管教,你們要忍受,因為 神待你們好像待兒子一樣;哪有兒子不受父親管教的呢? \\ \\
If you are left without discipline, in which all have participated, then you are illegitimate children and not sons. \\
作兒子的都受過管教。如果你們沒有受管教,就是私生子,不是兒子了。 \\ \\
Besides this, we have had earthly fathers to discipline us and we respected them. Shall we not much more be subject to the Father of spirits and live? \\
還有,肉身的父親管教我們,我們尚且敬重他們;何況那萬靈的父,我們不是更要順服他而得生嗎? \\ \\
For they disciplined us for a short time at their pleasure, but he disciplines us for our good, that we may share his holiness. \\
肉身的父親照著自己的意思管教我們,只有短暫的日子;唯有 神管教我們,是為著我們的好處,使我們在他的聖潔上有分。 \\ \\
For the moment all discipline seems painful rather than pleasant; later it yields the peaceful fruit of righteousness to those who have been trained by it. \\
但是一切管教,在當時似乎不覺得快樂,反覺得痛苦;後來卻為那些經過這種操練的人,結出平安的果子來,就是義。 \\ \\
Therefore lift your drooping hands and strengthen your weak knees, \\
所以,你們要把下垂的手和發軟的腿挺直起來; \\ \\
and make straight paths for your feet, so that what is lame may not be put out of joint but rather be healed. \\
也要把你們所走的道路修直,使瘸子不至於扭腳,反而得到復原。 \\ \\
Strive for peace with all men, and for the holiness without which no one will see the Lord. \\
你們要竭力尋求與眾人和睦,並且要竭力追求聖潔。如果沒有聖潔,誰也不能見主。 \\ \\
See to it that no one fail to obtain the grace of God; that no "root of bitterness" spring up and cause trouble, and by it the many become defiled; \\
你們要小心,免得有人失去了 神的恩典;免得有苦根長起來纏繞你們,因而污染了許多人; \\ \\
that no one be immoral or irreligious like Esau, who sold his birthright for a single meal. \\
又免得有人成為淫亂的和貪戀世俗的,好像以掃一樣,為了一點點食物,竟把自己長子的名分出賣了。 \\ \\
For you know that afterward, when he desired to inherit the blessing, he was rejected, for he found no chance to repent, though he sought it with tears. \\
你們知道,後來以掃想要承受祝福,卻被拒絕了;他雖然帶著眼淚尋求,還是沒有反悔的餘地。 \\ \\

\hline
\end{tabularx}

\newpage

\begin{tabularx}{\textwidth}{p{0.95\textwidth}}
\hline
Hebrews Ch. 12 continue \\
\hline \\
For you have not come to what may be touched, a blazing fire, and darkness, and gloom, and a tempest, \\
你們不是來到那座摸得著的山。那裡有烈火、密雲、幽暗、暴風、 \\ \\
and the sound of a trumpet, and a voice whose words made the hearers entreat that no further messages be spoken to them. \\
號筒的響聲和說話的聲音;那些聽見這聲音的人,都請求 神不要再向他們多說話; \\ \\
For they could not endure the order that was given, "If even a beast touches the mountain, it shall be stoned." \\
因為他們擔當不起那命令:“就是走獸挨近這山,也要用石頭把牠打死。” \\ \\
Indeed, so terrifying was the sight that Moses said, "I tremble with fear." \\
當時,顯出的景象是那麼可怕,連摩西也說:“我非常恐懼戰兢。” \\ \\
But you have come to Mount Zion and to the city of the living God, the heavenly Jerusalem, and to innumerable angels in festal gathering, \\
你們卻是來到錫安山和永活的 神的城,就是天上的耶路撒冷;在那裡有千萬的天使聚集, \\ \\
and to the assembly of the first-born who are enrolled in heaven, and to a judge who is God of all, and to the spirits of just men made perfect, \\
有名字登記在天上眾長子的教會,有審判眾人的 神,有被成全的義人的靈魂, \\ \\
and to Jesus, the mediator of a new covenant, and to the sprinkled blood that speaks more graciously than the blood of Abel. \\
有新約的中保耶穌,還有他所灑的血。這血所傳的信息比亞伯的血所傳的更美。 \\ \\
See that you do not refuse him who is speaking. For if they did not escape when they refused him who warned them on earth, much less shall we escape if we reject him who warns from heaven. \\
你們要謹慎,不要棄絕那位說話的,因為從前的人棄絕了那位在地上警戒他們的,尚且不能逃罪;何況現在我們背棄那位從天上警戒我們的呢? \\ \\
His voice then shook the earth; but now he has promised, "Yet once more I will shake not only the earth but also the heaven." \\
當時他的聲音震動了地;現在他卻應許說:“下一次,我不但要震動地,還要震動天。” \\ \\
This phrase, "Yet once more," indicates the removal of what is shaken, as of what has been made, in order that what cannot be shaken may remain. \\
“下一次”這句話,是表明那些被震動的,要像被造之物那樣被除去,好使那些不能震動的可以留存, \\ \\
Therefore let us be grateful for receiving a kingdom that cannot be shaken, and thus let us offer to God acceptable worship, with reverence and awe; \\
因此,我們既然領受了不能震動的國,就應該感恩,照著 神所喜悅的,用虔誠敬畏的心事奉他; \\ \\
for our God is a consuming fire. \\
因為我們的 神是烈火。 \\ \\

\hline
\end{tabularx}

\newpage

\begin{tabularx}{\textwidth}{p{0.95\textwidth}}
\hline
Hebrews Ch. 13 \\
\hline \\
Let brotherly love continue. \\
你們總要保持弟兄的愛。 \\ \\
Do not neglect to show hospitality to strangers, for thereby some have entertained angels unawares. \\
不要忘了用愛心接待人,有人就是這樣作,在無意中就款待了天使。 \\ \\
Remember those who are in prison, as though in prison with them; and those who are ill-treated, since you also are in the body. \\
你們要記念那些被囚禁的人,好像跟他們一起被囚禁;也要記念那些受虐待的人,好像你們也親自受過。 \\ \\
Let marriage be held in honor among all, and let the marriage bed be undefiled; for God will judge the immoral and adulterous. \\
人人都應該尊重婚姻,婚床也不要玷污,因為 神一定審判淫亂的和姦淫的人。 \\ \\
Keep your life free from love of money, and be content with what you have; for he has said, "I will never fail you nor forsake you." \\
你們為人不要貪愛錢財,要以現在所有的為滿足;因為 神親自說過:“我決不撇下你,也不離棄你。” \\ \\
Hence we can confidently say, "The Lord is my helper, I will not be afraid; what can man do to me?" \\
所以我們可以放膽說:“主是我的幫助,我決不害怕,人能把我怎麼樣呢?” \\ \\
Remember your leaders, those who spoke to you the word of God; consider the outcome of their life, and imitate their faith. \\
你們要記念那些領導過你們,把 神的道傳給你們的人;你們要觀察他們一生的成果,要效法他們的信心。 \\ \\
Jesus Christ is the same yesterday and today and for ever. \\
耶穌基督昨天、今天、一直到永遠都是一樣的。 \\ \\
Do not be led away by diverse and strange teachings; for it is well that the heart be strengthened by grace, not by foods, which have not benefited their adherents. \\
你們不要被各樣怪異的教訓勾引去了。人心靠著恩典,而不是靠著食物得到堅定,才是好的;因為那些拘守食物的人,從來沒有得過益處。 \\ \\
We have an altar from which those who serve the tent have no right to eat. \\
我們有一座祭壇,壇上的祭物,是那些在會幕中供職的人沒有權利吃的。 \\ \\
For the bodies of those animals whose blood is brought into the sanctuary by the high priest as a sacrifice for sin are burned outside the camp. \\
那些祭牲的血,由大祭司帶進聖所作贖罪祭,祭牲的身體卻要在營外焚燒。 \\ \\
So Jesus also suffered outside the gate in order to sanctify the people through his own blood. \\
所以耶穌也是這樣在城門外受苦,為的是要藉著自己的血使人民成聖。 \\ \\
Therefore let us go forth to him outside the camp and bear the abuse he endured. \\
那麼,讓我們也出到營外到他那裡去,擔當他的凌辱。 \\ \\
For here we have no lasting city, but we seek the city which is to come. \\
因為在這裡我們沒有長存的城,我們卻是尋求那將要來的城。 \\ \\
Through him then let us continually offer up a sacrifice of praise to God, that is, the fruit of lips that acknowledge his name. \\
所以,我們要藉著耶穌,常常把頌讚的祭品獻給 神,這就是承認他的名的人嘴唇的果子。 \\ \\
Do not neglect to do good and to share what you have, for such sacrifices are pleasing to God. \\
你們也不要忘記行善和捐輸,這樣的祭是 神所喜悅的。 \\ \\
Obey your leaders and submit to them; for they are keeping watch over your souls, as men who will have to give account. Let them do this joyfully, and not sadly, for that would be of no advantage to you. \\
你們要聽從那些領導你們的人,也要順服他們;因為他們為你們的靈魂警醒,好像要交帳的人一樣。你們要使他們交帳的時候快快樂樂,不至於歎息;如果他們歎息,對你們就沒有好處了。 \\ \\

\hline
\end{tabularx}

\newpage

\begin{tabularx}{\textwidth}{p{0.95\textwidth}}
\hline
Hebrews Ch. 13 continue \\
\hline \\
Pray for us, for we are sure that we have a clear conscience, desiring to act honorably in all things. \\
請為我們禱告,因為我們深信自己良心無虧,願意凡事遵行正道。 \\ \\
I urge you the more earnestly to do this in order that I may be restored to you the sooner. \\
我更加要求你們為我們禱告,好使我們能夠快點回到你們那裡去。 \\ \\
Now may the God of peace who brought again from the dead our Lord Jesus, the great shepherd of the sheep, by the blood of the eternal covenant, \\
願賜平安的 神,就是那憑著永約的血,把群羊的大牧人我們的主耶穌,從死人中領出來的那一位, \\ \\
equip you with everything good that you may do his will, working in you that which is pleasing in his sight, through Jesus Christ; to whom be glory for ever and ever. Amen. \\
在一切善事上成全你們,好使你們遵行他的旨意;又藉著耶穌基督在我們裡面,行他所喜悅的事。願榮耀歸給他,直到永永遠遠。阿們。 \\ \\
I appeal to you, brethren, bear with my word of exhortation, for I have written to you briefly. \\
弟兄們,我勸你們耐心接受我這勸勉的話,因為我只是簡略地寫給你們。 \\ \\
You should understand that our brother Timothy has been released, with whom I shall see you if he comes soon. \\
你們要知道,我們的弟兄提摩太已經釋放了;如果他來得早,我就跟他一起去看你們。 \\ \\
Greet all your leaders and all the saints. Those who come from Italy send you greetings. \\
請問候所有領導你們的人和所有聖徒。從意大利來的人也問候你們。 \\ \\
Grace be with all of you. Amen. \\
願恩惠與你們眾人同在。 \\ \\

\hline
\end{tabularx}


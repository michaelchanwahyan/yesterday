\section{``YESTERDAY"}
Perhaps no word expresses the thought framework of Hebrews so well as does
``yesterday" (ἐχθές); no word serves better to prevent a false understanding of
the author's viewpoint.
This word occurs in 13.8, and in the RSV this verse is translated: ``Jesus
Christ is the same yesterday and today and for ever."
More Literally the Greek may be translated: ``Jesus Christ yesterday and today
the same and into the ages."
At first sight this verse seems to stand alone, lacking any thought connection
with either v.7 or v.9.
But this impression is misleading.
\newline

Verse 7 has spoken of the former leaders of the Christians addressed.
They were exemplary not only in preaching the gospel but in their faithfulness
(in Hebrews the word `faith' includes a strong note of faithfulness).
Their faithfulness was maintained to their death, to which the author refers
when he speaks of `the outcome' of their life.
The recipients of Hebrews, when they think of the faithfulness of their ioneer
leaders, should be encouraged to maintain a like faithfulness, even if, after
their earlier hardship, they must again suffer persecution.
\newline

But the reason for such faithfulness is not merely that they should imitate the
steadfastly loyal life of their former leaders.
They can and should look constantly to Jesus Christ, `the pioneer and perfecter
of our faith' (12:2).
He is the supreme example of faithfulness and constancy, `the same yesterday and
today and for ever'.
If they keep clearly in view the unswerving loyalty and steadfastness of Jesus
Christ, they will `not be led away by diverse and strange teachings' such as v.9
warns against.
Verse 8 is thus not an isolated and unconnected remark; it connects with both
what precedes and what follows.
\newline

But our point here is more general.
This verse is notably suited to serve as a true guidepost to the view point of
Hebrews.
In particular, the word `yesterday', properly understood, summarizes and points
to the distinctive view point and message of the author.
It points to the view that lies behind and finds frequent expression in chapter
1-12.
\newline

This view has at times been missed, particularly by interpreters who have been
misled by a wrong use of the words `the same'.
Under the influence of this phrase, it has sometimes been thought that 13:8 is
emphasizing the unchanging nature of Jesus Christ.
This seems to many Christians a most reasonable interpretation, for if Jesus
Christ is truly and fully divine, he would not be expected to be subject to
change; God, as the Westminster Shorter Catechism confesses, in the answer to
Question 4, is `unchangeable'.
\newline

To this argument from the unchangeable nature of God is often added, usually
more unconsciously than by conscious decision, the assumption that reality is
timeless.
Created things have no permanent existence; they may and will change and pass
away.
But what is true and real and ultimate belongs to that timeless realm which
alone has primary significance.
As a result of this viewpoint many interpreters of Hebrews find a dominantly
Platonic note in Hebrews, and the phrae `the same' is to them a reference to the
timeless and so ultimately real significance of Jesus Christ.
\newline

This puts before us a basic question.
What for the writer of Hebrews was the nature of ultimate reality?
Was it time with its sequence of historical events as the key to a sound
understanding of life?
Or was it space as a symbol of two realms of life, one of which -- and that a
secondary one -- was this earthly time-bound order, while the other -- and that
the truly significant one -- was the transcendent order?
In the centuries of New Testament interpretation there has been a persistent
tendene to regard the writer of Hebrews as essentially Platonic in his outlook
and thought.
To him, it is often supposed, time with its sequence of events is not the proper
or adequate form in which to state the essential Christian gospel.
\newline

But 13:8 does not embody a basically Platonic point of view.
The key word that points away from an essentially Platonic, basically timeless
manner of thoughts is `yesterday'.
To be sure, if as has been suggested we were to take this word to refer to all
previous time and so to mean `from all eternity', the ideas of time and change
would not be dominant.
But `yesterday' does not mean `from all eternity' or `throughout all the vast
vistas of preceding time'.
It points to Jesus Christ as one who just recently became what he now is and
what he always will be in all the endless succession of future ages.
\newline

This does not imply that quite recently Jesus Christ became completely other
than what he had been before.
He was and is and will remain the divine Son.
But he was not until recently the qualified high priest who could make the
once-for-all and fully effective sacrifice.
\newline

This may seem to us a shocking statement, but it is basic to a true
understanding of Hebrews, whose author asserts it with unmistakable clarity:
Christ `{\it learned} obedience through what he suffered; and {\it being made
perfect} he became the source of eternal salvation to all who obey him' (5:8-9).
To offer the perfect sacrifice he had to be the perfect high priest.
To be the qualified perfect high priest he had to learn obedience through
suffering in his earthly life.
To achieve this he necessarily `for a little while was made lower than the
angels' (2:9) and `had to be made like his brethren in every respect, so that he
might become a merciful and faithful high priest in the service of God, to make
expiation for the sins of his people' (2:17-18).
\newline

Amazing as the idea may seem to men, God had to `make the pioneer of their
salvation perfect through suffering' (2:10).
To enter the true and heavenly sanctuary, to offer there the one perfect
sacrifice on our behalf (9:11-14), and to intercede there for his people (7:25),
Jesus had to become qualified, to be perfected, to learn obedience through what
he suffered, to present his own blood as the perfect sacrifice, and then to
reign with God (1:3) and continue his high priestly ministry by his intercession
for his people (7:25).
\newline

All this, the author attests, has taken place.
It has taken place `in these last days' (1:2), and while the author is conscious
of belonging to at least the second generation of Christians (2:3; 13:7), this
does not prevent him, as he sees the coming and work of Christ in the long sweep
of God's dealings with Israel, from speaking of these decisive events as having
occurred `yesterday'.
\newline

In this drama of redemption time is real; it is the setting for the saving work
of Christ.
And specific unique events are decisive.
It was in crucial divine acts of God that salvation was won and made available.
This gospel has to be stated in terms of time, in terms of one decisive network
of historical events.
\newline

Nothing like it -- except by way of a type or foreshadowing of that unique and
fully effective climactic working of God in Christ -- had been seen or done
before.
Nothing to rival it can ever be expected again, for the work of Christ is
complete and fully adequate and will never need to be done again.
No new change in Jesus Christ will be expected or needed, for he has been `made
perfect' (5:9).
So Jesus Christ is `the same' `today' as he became `yesterday', and he will
remain `the same' on into the endless ages to come, `for ever'.
\newline

This sense of the decisive and permanent effect of a unique recent historical
event the author expresses by the use of two synonymous adverbs, `once'
(ἄπαξ) and `once for all' (ἐφἀπαξ).

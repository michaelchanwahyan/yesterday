\section{``YESTERDAY"}
Perhaps no word expresses the thought framework of Hebrews so well as does
``yesterday" (ἐχθές); no word serves better to prevent a false understanding of
the author's viewpoint.
This word occurs in 13.8, and in the RSV this verse is translated: ``Jesus
Christ is the same yesterday and today and for ever."
More Literally the Greek may be translated: ``Jesus Christ yesterday and today
the same and into the ages."
At first sight this verse seems to stand alone, lacking any thought connection
with either v.7 or v.9.
But this impression is misleading.
\newline

Verse 7 has spoken of the former leaders of the Christians addressed.
They were exemplary not only in preaching the gospel but in their faithfulness
(in Hebrews the word `faith' includes a strong note of faithfulness).
Their faithfulness was maintained to their death, to which the author refers
when he speaks of `the outcome' of their life.
The recipients of Hebrews, when they think of the faithfulness of their ioneer
leaders, should be encouraged to maintain a like faithfulness, even if, after
their earlier hardship, they must again suffer persecution.
\newline

But the reason for such faithfulness is not merely that they should imitate the
steadfastly loyal life of their former leaders.
They can and should look constantly to Jesus Christ, `the pioneer and perfecter
of our faith' (12:2).
He is the supreme example of faithfulness and constancy, `the same yesterday and
today and for ever'.
If they keep clearly in view the unswerving loyalty and steadfastness of Jesus
Christ, they will `not be led away by diverse and strange teachings' such as v.9
warns against.
Verse 8 is thus not an isolated and unconnected remark; it connects with both
what precedes and what follows.
\newline

But our point here is more general.
This verse is notably suited to serve as a true guidepost to the view point of
Hebrews.
In particular, the word `yesterday', properly understood, summarizes and points
to the distinctive view point and message of the author.
It points to the view that lies behind and finds frequent expression in chapter
1-12.
\newline

This view has at times been missed, particularly by interpreters who have been
misled by a wrong use of the words `the same'.
Under the influence of this phrase, it has sometimes been thought that 13:8 is
emphasizing the unchanging nature of Jesus Christ.
This seems to many Christians a most reasonable interpretation, for if Jesus
Christ is truly and fully divine, he would not be expected to be subject to
change; God, as the Westminster Shorter Catechism confesses, in the answer to
Question 4, is `unchangeable'.
\newline

To this argument from the unchangeable nature of God is often added, usually
more unconsciously than by conscious decision, the assumption that reality is
timeless.
Created things have no permanent existence;

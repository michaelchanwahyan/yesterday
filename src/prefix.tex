\chapter*{PREFACE}
The discussion that follows is by no means a complete study of Hebrews.
It does not pretend to deal with all questions that would be involved in a
relatively thorough study, nor does it atempt to present the full range of
available bibliography.
Those who want to see how varied and extensive are the recent studies of
Hebrews may turn to the important bibliographical and evaluative article by
Erich Grasser entitled `Der Hebraerbrief 1938-1963', published in
{\it Theologische Rundschau} 30, 1964, pp. 138-236.

The attempt we make in the present study is to find and vindicate a new
approach which will let us understand better the literary form, the key themes,
and the basic unity of Hebrews.
To put our purpose another way, we here try to let the author say what he wants
to say; we do not intend o make him fit into the patterns either of our modern
thought or of other New Testament writers.
He shows agreements with Christian writers of his time, but he is no mere echo
of or duplicate of any contemporary author.
It will be a great gain if we can hear him speak in his own way and let him put
the emphasis on his own key ideas and aims.
With great ability and skill he confronts us with the key issues of Christian
faith and thought, and the Church may listen with profit to what he has to say
to our later day.
\newline

\hfill~\texttt{FLOYD V. FILSON}
\newline
{\it Chicago}
\newline
{\it $\it 5$ October, $\it 1966$}
\newline

Note: Except in special cases the biblical quotations are taken by permission
from the Revised Standard Version.


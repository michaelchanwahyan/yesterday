\chapter{THE FORM AND FUNCTION OF CHAPTER 13}

\section{Unique Aspects of this Chapter}
The reader of Hebrews cannot fail to sense a change of tone and content when he
passes from the close of chapter 12 to the beginning of chapter 13.
Up to that point the argument has been closely knit and carefully developed.
Jesus Christ the divine Son is the unique High Priest whose once-for-all unique
and sufficient priestly sacrifice, followed by his continual priestly
intercession, provides the full answer to the spiritual needs of men.
The exhortation to take full advantage of this divinely provided way of
salvation and to live faithfully in obedience to the Lord is matched by the
urgent warning not to fall away from so great a privilege, `for our God is a
consuming fire' (12:19).

The unity of these twelve chapters is clear.
It is possible--though not probable--that in developing his argument in such
discussions as 1:5-13 and 2:6-13 th author of Hebrews used an already existing
`Testimony Book', a collection of related and particularly useful Scripture
passages.
It may be--though it need not be so--that the exposition and application of
certain Old Testament passges, namely, Ps. 2:7 (quoted in 5:5); Ps. 40:6-8 (in
10:5-7); Ps. 95:7-11 (ub 3:7-11); Ps. 110:4 (in 5:6), and Jer. 31:31-34 (in
8:8-12), and also in the extended discussion of faith in chapter 11, the author
of Hebrews made use of biblical expositions which he had previously developed
and had used in his teaching prior to the writing of Hebrews itself.
But to whatever oral sources or earlier literary stages the author may have
been indebted, the result is his own well-knit discussion, and there is no
reason to question its basic unity or the sincere urgence of its appeal.

The change of tone at 13:1 is unmistakable.
The crisis atmosphere is no longer felt.
The urgent exhortation to hold fast to the faith in a crisis situation has
suddenly given way to a varied series of imperatives, which present a wide
range of pastoral instruction.
Nearly twenty imperatives or exhortations occur in chapter 13; they deal with
some fifteen separate topics.
In general, the tone is one of varied pastoral counsel; it is not that of a
life-and-death crisis which limits attention to one crucial concern.
None of the items treated is discussed at length.

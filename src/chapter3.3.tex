\section{``Jesus Christ"}
Before attempting to state in any detail how the author of Hebrews thought of
Jesus, it was necessary to point out how decisive for this author was the
historical understanding behind the word `yesterday'.
In that discussion it became clear that the decisive historical event of which
Hebrews speak centred in a person, Jesus Christ.
It now is necessary to study more closely the identity and work of that person.
\newline

It becomes clear at once that chapter 13 does not parallel all the titles of
Jesus and all the descriptions of his work which appear in chapters 1-12.
But this could not be expected.
Chapter 1-12 present the person and work of Jesus as the basis for the author's
effective `word of exhortation' (13:22).
Chapter 13, as we saw when we made clear its fourfold structure, is largely
given over to a wide-ranging list of general exhortations and personal message.
It could not be expected to repeat all that chapters 1-12 have said about Jesus
and his work.
But if we first survey the titles used of Jesus in chapters 1-12 and then ask
how far the references in chapter 13 parallel that picture, we shall find as
wide a range of agreement as we have any right to expect.

\begin{enumerate}[leftmargin=0pt,
                  itemindent=20pt,
                  labelwidth=15pt,
                  labelsep=5pt,
                  listparindent=0.7cm,
                  align=left,
                  label=(\roman*)]
    \item {\it Jesus}. One fact should be noted at the outset.
          The author does not use Melchizedek as a name for Jesus.
          He calls Jesus a high priest `after the order of Melchizedek' (5:6,10;
          etc.) and `in the likeness of Melchizedek' (7:15), but he does not use
          Melchizedek as a name or title of Jesus.
          And although his method of using Scripture might lead us to expect him
          to call Jesus `King of Righteousness' and `King of Peace; (7:3), he
          never does so.
          It is not Melchizedek's position as king but his priestly position and
          role which interests the author; the superiority of the priestly
          ministry of Jesus as compared with the ministry of the Levitical
          priesthood is the point he draws from Melchizedek's superiority to
          Abraham (7:4-10).
          \newline

          Another fact worth notice is that the author does not find
          significance in the root meaning of the name and titles of Jesus.
          The word Jesus is the Greek form of the Hebrew name Joshua and means
          `Yah(weh) is salvation' or `Yah(weh) saves'.
          The writer of Matt. 1:21 was aware of that meaning when he ascribed to
          an angel the words, `You shall call his name Jesus, for he will {\it
          save} his people from their sins.'
          The author of Hebrews either did not know the root meaning of the
          Hebrew name or chose to make no use of it.
          He never uses the title Saviour or Redeemer of Jesus.
          He thinks of Jesus as the historical person who acted deisively for
          God to make `purification for sins' and provide `eternal salvation'
          (1:3; 5:9), but he does not draw this conclusion from the root meaning
          of the name.
          \newline

          Similarly there is no evidene that he knew and used the root meaning
          of the word Christ, Χριστὀς, means `anointed'.
          The author uses the verb χρἰω once(1:9), but he does not connect it
          with the messianic meaning it had in the early Apostolic Age; he
          speaks in 1:9 of the divine Son, whom God `has anointed... with the
          oil of gladness beyond thy comrades', that is, God has anointed or
          endowed the Son with a higher degree of joy and bliss than is enjoyed
          by his comrades, whether angels or men.
          There is no indication that in his rather frequent use of the word
          Christ the author thought of the word's root meaning or of its earlier
          messianic referene.
          To him the word Christ was no longer an adjective or title but had
          become a proper name, especially in the thrie used double name `Jesus
          Christ' (10:10; 13:8, 21).
          \newline
          
          A third general remark is that the titles used of Jesus are not
          mutually exclusive.
          The author does not ue one title for one period of Jesus' work while
          using another title of another period.
          The titles overlap in their range of referene.
          This will become clear as we now survey the chief ways in which the
          author of Hebrews designates or describes Jesus.
          \newline

          The basic function of the name Jesus is to designate the central
          figure of the gospel story, the historical figure whose work
          `yesterday' was God's decisive redemptive act `in these last days'
          (1:2).
          Little is said in detail of his life, though more is said than is
          sometimes supposed.
          He became man (2:5-18), `like his brethren in every respect' except
          sin (2:17; 4:15).
          This implies pre-existence and incarnation and entrance upon a form of
          life `lower than the angels' (2:9).
          He had to `share in flesh and blood' (2:14) to become able to help
          men, but the `How' of the incarnation is not discussed.
          It was necessary for him to have `suffered and been tempted' to be
          `able to help those who are tempted' (2:18).
          He had to withstand temptation; he can help men only because he `in
          every respect has been tempted as we are, yet without sinning' (4:15).
          \newline

          Hebrews is the one New Testament writing outside the Synoptic Gospels
          which frankly and clearly states that Jesus was tempted.
          But it differs from the Synoptics as to the time and function of the
          temptation.
          In the Synoptics Jesus is tempted after his baptism and before the
          opening of his public ministry (Mark 1:9-15); the Gethsemane struggle
          (Mark 14:32-42) is not clearly treated as his climactic temptation.
          In Heb. 5:7-9, however, it is the Gethsemane struggle that seems
          chiefly and clearly in mind, though this is not explicitly stated.
          In that crucial test he `learned obedience through what he suffered'
          and by his faithfulness under temptation he was `made perfect' and so
          was competent to act decisively to save and purify men.
          This salvation `was declared at first by the Lord' Jesus (2:3); the
          author of Hebrews knows that Jesus had a teaching ministry.
          But this was not the author's real interest; it was the human
          Gethsemane struggle wherein Jesus learned to be fully obedient under
          the test of suffering which was of crucial importance.
          The maturity of victory over temptation was the necessary
          qualification for his real ministry.
          \newline

          So in a sense Jesus' earthly life and ministry and even his acceptance
          of death, when he `suffered outside the gate' of Jerusalem (13:12),
          were only necessary--but really necessary--preliminaries for his
          effective high priestly ministry.
          The author of Hebrews, as we shall understand better when we come to
          our next section on Jesus' sacrifice, both magnifies and minimize
          Jesus' earthly life and temptation and physical death.
          \newline

          It is a good clue to the centre of interest of this author that in all
          his discussion of the death of Jesus he never explicitly refers to
          Jesus' resurrection.
          His thought hurries on past the earthly triumph over physical death,
          which is only implied in 13:20, to the heavenly ministry and triumph.
          For the statement that `God... brought again from the dead our Lord
          Jesus' means something more than the rising of Jesus to renewed life
          on this earth; it includes and mainly concerns the bringing of Jesus
          into the heavenly setting where his real priestly ministry is carried
          out.

    \item {\it Christ}. The title Christ, as has been noted, is used without any
          clear reference to the traditional messianic role and functions.
          The author knows that Jesus `was descended from Judah' (7:14), and so
          in his discussioon he might have made use of the expectation of a
          Davidic Messiah.
          But he makes nothing of this Davidic connection and his use of the
          title Christ has much the same sweep as have the referenes to Jesus.
\end{enumerate}

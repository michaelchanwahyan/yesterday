\section{Is Chapter 13 An Integral Part Of Hebrews?}
Once we face clearly the different content, tone, and form of this chapter, we
are not surprised that some scholars have challenged its authenticity.
The question has been raised whether chapter 13 is in whole or in part an
addition to the original writing.

E.D. Jones, in an article on `The Authorship of Hebrews xiii', made the
improbable claim that Hebrews 13 was the closing part of Paul's `severe letter'
to Corinth.
Jean Hering regarded chapter 13:1-21 as a letter which the writer of the
original homily (chapter 1-12) added when he sent a copy of his work to a
specific local church.
13:22-25, Hering thought, may have been added by another hand, perhaps by Paul
if Apollos wrote the reset of Hebrews in the sixties.

Other scholars have considered only a part of chapter 13 to be an addition to
the original work.
C. C. Torrey, for example, considered 13:8-15 to be a later addition.
More than one scholar has been led by the formal benediction in 13:20-21 and
by the `postscript' character of 13:22-25 to conjecture that these four closing
verses of chapter 13 were added to the original work, which ended with 13:2-21.
The view of Hering was noted in the previous paragraph.
F. J. Badcock regarded 13:23-25 as a postscript added by Paul to a letter that
was `the voice of Barnabas' by `the hand of Luke'.
A. Vanhoye, who claimed to find in Hebrews a carefully constructed chiastic
structure, repeatedly woven together by key words which occur at the beginning
of a subdivision and then appear again at or very near to the close of the
section, found that his theory could not include 13:19 or 13:22-25 in its neat
chiastic structure.
He conjectured that these verses were added to the original work when it was
being sent to a particular church.

W. Wrede noted a difference between the earlier chapters of the writing and the
closing chapter 13.
He conjectured that the writer set out to produce a formal treatise, but as he
came to the close of his work he decided to give it the appearance of a letter
by Paul; he did not revise the earlier part of his writing to conform to this
altered purpose, but contented himself with making the ending hint at the
Pauline authorship which he hoped would give acceptance and influence to his
writing.

None of these critical views has won any wide acceptance.
In fact, it is a little surprising, in view of the decidedly different tone and
content of chapter 13, that so few scholars have felt driven to such hypotheses.
The steadily dominant view that Hebrews can best be explained as a literary
unity is represented by the strong arguments for unity by R. V. G. Taasker and
C. Spicq.

To support such a conclusion and add fresh reasons for regarding Hebrew 13 as an
original and integral part of the writing, we must first survey the varied views
as to the literary form of Hebrews.
Against that background we can show how chapter 13 fits effectively into the
form used.

\newpage
\section{The Relation of Chapter 13 to Chapter 1-12}
Once we have grasped this fourfold structure of chapter 13, we can see how to
study this chapter in relation to the earlier chapters of Hebrews.
Chapter 1-12 obviously have a well-knit structure of thought which we cannot
expect to find in the wide-ranging pastoral concern and personal items of
chapter 13.
The literary and topical unity of chapters 1-12 will not be matched by the more
general range of interests in chapter 13.
There may be aspects of Christian thought which are carefully developed in the
course of chaptes 1-12 but are not found at all in chapter 13.
Chapter 13 in turn may touch on Christian obligations and personal situations
which there was no reason to discuss or even mention in chapters 1-12.
We cannot expect to find in chapter 13 neat and complete parallels to all that
chapters 1-12 present.

If, however, the same man wrote the entire work as we have it in our New
Testament, we can expect to find in chapter 13 some measure of agreement with
chapters 1-12 in vocabulary, style, and content.
To some degree the themes and outlook of chapter 13 should show similarity to
the earlier chapters.
We propose to test this possibility.
We propose to study the themes and concerns of chapter 13 to see how far they
are paralleled in the themes and interests of chapters 1-12.

Such a study might take the form of a verse by verse examination of chapter 13,
an examination concluded by a summary and interpretation of the individual items
learned in such an exegetical study.
It should make for greater clarity and better focus, however, to direct
attention to the key themems that stand out in chapter 13 and ask in the case of
each one how far it is paralleled in chapters 1-12.
The result may be to throw new light on the earlier chapters as well as on the
meaning and importance of the closing chapter of Hebrews.
The parallels may be the more striking because they result from the comparative
study of two quite different forms of literary expression, the close knit
discussion of the first twelve chapters and the more general and wide-ranging
exhortation of chapter 13.

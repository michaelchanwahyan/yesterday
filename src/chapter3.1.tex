\chapter{KEY THEMES OF CHAPTER 13}

\section{``MY WORD OF EXHORTATION"}
\Large
In these words, as has already been noted, the author has given the most apt
(貼切) description that could be used to state briefly the nature and purpose
of his writing.
He has written to arouse, urge, encourage, and exhort (勸勉) the hearers of his
message to realize how completely God has met their need by the work of Christ
and what a priceless privilege (特權) they have thus been given.

He recognizes that in what he has said he has gone beyond ``the elementary
doctrines (教義) of Christ"
{\large (6.1 所以,我們應當離開基督初步的道理,努力進到成熟的地步,
不必在懊悔致死的行為,信靠 神,洗禮,按手禮,死人復活,
和永遠審判的教訓上再立根基。)}
and given the Christians addressed ``solid food" such as is suited ``for the
mature" members of the Church
{\large (5.14 只有長大成人的,才能吃乾糧,他們的官能因為操練純熟,
就能分辨是非了。)}.
The addressees have failed thus far to attain the Christian maturity which in
view of their years of discipleship (作為門徒) should now mark their thought
and conduct
{\large (5.12 到這個時候,你們應該已經作老師了;
可是你們還需要有人再把 神道理的初步教導你們。
你們成了只能吃奶而不能吃乾糧的人!)}.
But he is not content (滿足) to repeat the rudimentary (初步) teachings they
should long ago have mastered.
He chooses rather to go on to an advanced discussion of the high priesthood of
Christ and his once-for-all completely effective sacrifice for sins.

As he does this, his work could seem to be essentially a thorough theological
discussion.
But that would be a mistaken conception of his basic purpose.
His aim is not merely intellectual (知識性) and theological.
He uses these aspects to sweep aside any doubts concerning the truth and
crucial importance of the gospel, to make clear the completeness and adequacy
of God's work in Christ, and to drive home the fact that by their acceptance or
indifference (漠不關心) these Christians will make the crucial decision which
will determine their spiritual status now and for all time to come.
In every statement with important theological content he is urging the
recipients to realize how much is at stake (處於危險中) in their response to
this gospel.

To so basic and decisive a message mild adhesion (粘附) of lukewarm (溫和)
allegiance (忠誠) is no adequate response.
Nothing less is called for than grateful faith expressed in active, steady
faithfulness.

But the author does not express his urgency merely by the urgent tone in which
he presents the truth of the gospel.
He repeatedly follows a passage of earnest (嚴謹) theological discussion by
explicit and extended exhortation to take seriously and act upon the truth he
has presented.
This vibrant hortatory note occurs not only in 2.1-4; 3.7-4.13; 4.14-16;
5.11-6.12; 10.19-39; and 12.1-29 but also in the greater part of chapter 13,
which thus has the same dominant note as do the earlier chapters.
The urgency built into the theological discussion and the repeated explicit
exhortation which continually recurs in this series of exhortations cannot be
disregarded.
The note of exhortation is so recurrent and is expressed at such length as to
be dominant in the total work.

This writing is thus the author's ``word of exhortation" (勸勉的話) to the
Christians who will hear his message read to them in their assembly for common
worship.
His theological discussion is not an end in itself.
It is only his means to reach these Christians with an urgent exhortation to
hold fast to their faith, to show their faith in faithfulness, and to let no
passage of time or renewal of persecution (迫害) dull the stout (嵩厚)
steadfastness (堅定) of their Christian confession (懺悔) and obedience.

There have been attempts to deny that the author himself described his work as
his ``word of exhortation".
Overbeck, for example, held that 13.22-25 was a later addition, to promote the
(false) idea that Paul had written the work.
Wrede held a similar view, but with the idea that the author included these
closing words in order to suggest subtly Pauline authorship and so promote
acceptance of his work as from the Apostle Paul.
It has also been suggested that the ``word of exhortation" referred only to
chapters 1-12; in this case all of chapter 13 would be an addition to the
original work.
Vanhoye held that these closing verses (13.19,22-25) were a brief covering
letter; the ``word of exhortation on his view consisted only of 1.1-13.21
(less 13.19).

Once we have understood the fourfold structure of chapter 13 and have come to
see clearly that to follow the formal benediction of 13.20-21 with personal
messages and a shorter benediction conforms to a well-known practice, we have
no reason to feel uneasy concerning the presence of 13.22-25 in Hebrews.
In particular, the writer's informal but apt designation of 1.1-13.21 as his
``word of exhortation" is entirely in place and is suited to the context.

Since ``exhortation" (παράκλησις) is thus so apt a description of Hebrews, it
will be worth while to examine the word a little more closely.
It goes back to the verb παρακαλέω, which means ``beseech" (懇求),
``entreat" (求告), ``console" (寬慰), ``comfort", ``encourage", ``exhort".
Present in all these meanings is the idea of an urgent appeal (呼籲); the
various English translations reflect varying situations in the relation of the
person speaking to the person(s) addressed.

The noun παράκλησις occurs in the New Testament twenty-four times; we find it
four times in Acts (4.36; 9.31; 13.15; 15.31), seventeen times in eight Pauline
letters (one is I Tim. 4.13), and three times in Hebrews (6.18; 12.5; 13.22).
The shade of meaning in each passage calls for study, especially since our
English words used to translate it have varied in meaning from one period to
another.
We can see this if we compare the English translations in the Authorized (King
James) Version (AV) with those in the Revised Standard Version (RSV).
The AV uses ``consolation" eleven times, ``comfort" four times, ``exhortation"
eight times and ``entreaty" once.
The RSV uses ``comfort" nine times (six of them in II Cor. 1.3-7),
``encouragement" six times, ``exhortation" five times, ``preaching" once
(II Cor. 8.4, where ``earnestly" more literally translated would be ``with much
earnestness"; AV has ``with much entreaty").

The dominant use of ``consolation" and ``comfort" in the AV is misleading today.
These words now cary to much of a passive note; they suggest mainly getting
perople to accept with resignation and submission whatever sorrow or tragedy
has come to theme.
In AD 1611 there was heard in these words, particularly in ``comfort", more of
a note of strengthening, encouraging, and summoning the afflicted (折磨) and
hard pressed to live with new courage and vitality in view of the gifts,
resources, and tasks which God bestows (賜給).
Even ``console", as Webster's Third New International Dictionary says of its
use today, meant not only to ``alleviate (緩和) the grief (哀痛)" but also to
``raise the spirits".
This encouraging note must be heard in the New Testament uses of the word
παράκλησις; it does note exclude the place of comforting and consoling in our
popular modern sense, but it subordinates that use to the more active meaning.

It would be well to avoid the translation ``comfort" and use ``encouragement"
and ``exhortation", not in order to eliminate the idea of soothing ``comfort"
in hardship and affliction, but to gather up this idea in the larger concept of
leading Christians to respond to sorrow, hardship, and affliction with
God-given strength and courage.
Such strength and courage are the result of the encouragement the gospel gives
and the exhortation it presents to live actively in faith that the gospel is
indeed dependably true.

Certainly in Heb. 13.22 the noun παράκλησις means the author's ``exhortation"
to believe and stake all on the truth of the gospel as he has set it forth.
It is an exhortation that throbs (感動) with ``encouragement" (though also with
stern (嚴厲) warning) and with earnest ``appeal".

Into such a ``word of exhortation" chapter 13 fits well.
It contains a series of imperatives, urgent exhortations, and statements of
personal concern for the Christians addressed.
Its content and hortatory (勸告) tone are in healthy harmony with the
courageous and encouraging attitude which the author has taken in the preceding
twelve chapters of Hebrews.


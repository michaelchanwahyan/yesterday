\newpage
\section{The Fourfold Structure of Chapter 13}
We noted earlier the at first sight rather curious fourfold structure of chapter
13.
After the carefully woven argument of chapters 1-12, focused on a right
understanding of Jesus Christ the one high priest and his once-for-all and fully
effective sacrifice, we suddenly encounter in chapter 13 a wide-ranging series
of imperatives and exhortations which are not directly connected withthe
preceding discussion.
We are no longer troubled y this sucession of general and quite vcaried commands
and instructions, for we have noted that the writer, out of a deep pastoral
concern and in view of his necessary absence from these Christians, takes this
opportunity to speak not only concerning the special problem discussed in
chapters 1-12 but also concerning other points which hethinks he should mention
for the good of the Christians addressed.
His inclusion of these added points expresses his pastoral concern for his
Christian friends and his awareness that they face more than one problem
concerning which they need pastoral instruction and exhortation.
\newline

When he has called such varied instructions to their attention and has given
them items of personal information, he shows by the formal benediction his
awareness that he is speaking through this writing to a congregation in common
worship.
It expresses his prayer for the effective working of God in them as they respond
faithfully to God's gifts by a life of fruitful helpfulness.
But though this would seem to be the fitting place to close the writing, the
sense of separation from the recipients and the sense of oneness with them in a
common bond of fatih and Christian concern leads him to add brief personal words
before he finally closes his writing by a shorter benediction which puts all
their personal relations and problems under the needed grace of God.
\newline

This fourfold pattern for the closing part of a written message--(1) varied
teaching, injunctions and information; (2) formal benediction; (3) personal
greetings and messages; and (4) closing brief benediction--may not be the way we
would close such a writing to Christian friends.
But before we conclude that the writer of Hebrews was a blundering composer who
could not follow accepted patterns of communication to Christian churches, there
is a highly important point to note about this fourfold structure.
Its use here is not unique in early in early Christian `letters'.
On the contrary, it is a pattern to which we find several parallels in the New
Testament.
Once we clearly see this, we shall not be tempted to belittle the literary
ability of he writer or think that chapter 13 must have been written by another
person than the author of chapters 1--12.
\newline

Let us note briefly other New Testament examples of this fourfold structure:
\newline

(i) {\it I Thessalonians}.
(a) In 5:12-22 we find a compact series of instructions concerning Christian
duty in a variety of life situations.
(b) Then follows in 5:23-24 a formal benediction and an assurance of God's
faithfulness.
(c) This is followed in 5:25-27 by a request for prayer for Paul, a concise
greeting to all, and instruction that the letter be read to all the Thessalonian
Christians.
(d) Then the writing concludes in 5:28 with a shorter benediction than the
formal one we found in 5:23-24.
\newline

(ii) {\it II Thessalonians}.
(a) A section of instructions and warnings closes with 3:15.
(b) Then 3:16 expresses an earnest prayer that the Thessalonian Christians may
be blessed with the peace and presence of God.
(c) Paul then takes the pen to add in 3:17 a personal greeting whose handwriting
will identify any letter  from him.
(d) Finally we have in 3:18 the fourth part of the fourfold structure, the brief
benediction prayer that God's grace may be with all the Thessalonian brethren.
\newline

(iii) {\it Galatians}.
(a) After a section of varied instructions concerning Christian living (6:1-10),
Paul takes the pen to summarize in 6:11-15 the message of this letter.
(b) Then 6:16 offers the earnest prayer that God's peace and mercy may rest upon
his faithful and true Israel, the Church.
But this is not the end of the letter, for
(c) 6:17 adds personal words, and
(d) they in turn are followed by the final benediction in 6:18.
\newline

(iv) {\it II Timothy}.
Here again the fourfold structure appears rather clearly.
(a) 5:9-18a contains personal information and instructions.
(b) A doxology follows (5:18b).
(c) Further personal information is given in 4:19-21.
(d) Then comes a benediction in two parts, praying that God's presence and grace
may be with Timothy and his companions in the churches he serves (`you' is
plural here, as it is in the closing verse of Titus and probably of First
Timothy).
\newline

(v) {\it Philippians}.
Here too the fourfold structure may be discerned.
(a) 4:10-19 deals with Paul's response to the gift the Philippian church has
sent to him.
(b) Then follows a doxology in 4:20.
(c) Personal greetings are added in 4:21-22, and
(d) the final benediction follows in 4:23.
\newline

(vi) {\it I Peter}.
The fourfold pattern is not so clear here.
But it is worth nothing that
(a) following various exhortations in 5:1-10 we find
(b) a formal ascription of praise to God and an `Amen' in 5:11,
(c) personal information and greetings in 5:12-14a, and
(d) a brief benediction in 5:14b.
The key feature of the fourfold structure, a formal benediction or doxology
followed by personal items and a brief closing benediction, is present.
\newline

(vii) {\it Romans}.
It would take us too far afield to discuss in detail all the problems raised by
the unusual number of benedictions found in the closing chapters of Romans.
In some manuscripts the elaborate doxology usually found in 16:25-27 is found at
the end of chapter 14, or at the end of chapter 15, or in two of these three
locations.
A short benediction is found in 15:33, in 16:20, and in the textually highly
suspect 16:24.
Since it is often suggested that chapter 16, or most of what it now contains,
was a separate letter of recommendation of Phoebe and was sent to Ephesus, and
since  there are indications that Romans in slightly shortened and edited forms
was circulated in the ancient Church as a statement of Paul's essential message,
it is difficult to know what to make of this abundance of benedictions, and we
cannot be certain exactly how Paul's original Letter to the Roman Church ended.
\newline
We need only say for our present purpose that this Letter too may have had the
fourfold structure found at the close of several Letters, in which two
benedictions formed a frequent though by no means ever-present feature of the
conclusion of New Testament letters.
The main letter was concluded by personal information and instructions followed
by a formal benediction or doxology, and this was followed by further personal
information and greetings and then by a final benediction.

In the other New Testament `letters' just examined we find so many parallels or
partial parallels to the fourfold structure of Hebrews 13 that there is no
reason for suspecting the unity and integrity of Hebrews because it has this to
us strange structure.
General instructions and information not closely connected with the content of
the main body of the writing were entirely in place.
The first Benediction showed the writer's awareness that his message would be
read aloud at a meeting of the congregation(s) to which it was sent.
The formal benediction was followed by personal items of various kinds.
Then a second benediction brouht the entire writing to a close.
We need have no hesitation in accepting Hebrews 13 as an integral part of the
total work.

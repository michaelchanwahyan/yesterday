\section{The Literary Form of Hebrews}
The difficulty in identifying the literary form of Hebrews is that none of the
terms we use to describe modern types of literary works fits exactly the form
and nature of Hebrews.
We naturally try to describe such a writing by one English word.
But in so doing we are inevitably led to use an inexact term, and so we conceal
to some degree the actual nature of the writing we are studying.
It will further our study if we list the chief terms used to describe the
literary nature of Hebrews and note the truth and the misleading implications of
each term.

(i) Hebrews has been called an essay or treatise.
These terms point to the serious, orderly, scholarly treatment of a theme, and
to some extent Hebrews fits this description.
But they fail in at least two respects to represent the nature and literary form
of this writing.
For one thing, a treatise or essay is a general discussion of some aspect of
truth and life, but  Hebrews was directed to a definite group of Christians and
concerned their urgent life situation.
It was to be read aloud to hat group to help it meet its crisis.

Moreover, a treatise or essay is content-centred; it aims to clarify truth.
But Hebrews is marked by repeated and urgent exhortation directed to the special
group addressed.
This personal focus and this hortatory tone are not adequately expressed by such
terms as treatise or essay.
The author's main attention is directed to the life situation of the people
addressed.
The concern of the writer is to guide those addressed to act loyally and
responsibly in the face of that situation.

(ii) The word oration has been used, though rarely, to describe Hebrews.
This term takes account of the fact that Hebrews was written to be read aloud to
a definite group of Christians.
But in itself the word oration fails to express the basic fact that this writing
was prepared to be read to a group from whom the writer is separated at the
time.
The written document is to some extent a substitute for an oral message directly
spoken to the people whom the author would like to address in person.
Someone else must read this writing to the people addressed.
The word oration does not express this fact.

(iii) Hebrews has often been called a sermon or homily.
In many ways this term is useful and accurate.
It expresses the personal concern of the writer for the spiritual welfare and
faithful integrity of the people addressed.
It indicates that this is a biblically-based Christian message which has an
assembled congregation in mind (or, if it is to be read to more than one
congregation in the city of its destination, it has all of these assembled
congregations in mind).
In a sermon or homily the note of exhortation, of urgent appeal, is inevitably
present, as it is in Hebrews.

Yet in at least two respects the word sermon or homily is not a correct and
adequate description of Hebrews.
In the first place, a sermon properly speaking is the direct personal statement
of Christian truth and the exhortation to the hearers to heed this truth.
It is of the essence of a sermon that the preacher faces his congregation and
speaks in person what he believes is the urgent word of God to them at the time.
But Hebrews is sent from a distance to be read to a congregation in the absence
of the author.

Moreover, the widely ranging general imperatives inchapter 13 are not what we
expect in a sermon, which normally comes to a conclusion and climax in the clear
statement and application of the aspect of truth on which the sermon centres.
One does not expect a sermon to conclude with a miscellaneous medley of
commands, personal information, and general instructions.
Hebrews is something more than and different from a sermon or homily.

(iv) Emphasis is often placed on the fact that Hebrews consists largely of
biblical exposition.
It is true that a considerable portion of the whole work cites and applies Old
Testament passages in the wording found in the Greek Old Testament.
The clear assumption of the writer is that the Old Testament foreshadows and
finds its fulfilment in the coming and work of Jesus Christ, the divine Son and
one effective high priest.
The concentration on the scriptural basis of the gospel message is so strong
that the discussion of the high priest and the sanctuary deals entirely with the
Tent in the wilderness as described in the Pentateuch; it does not deal at all
with the Temple in Jerusalem and the high priests who ministered there.

The fact remains, however, that the basic literary form of Hebrews is not
biblical exposition.
Substantial portions of the writing may be so described, but they use the
Scripture to throw light on the person and work of Jesus Christ, and a
significant portion of Hebrews is not cast in the form of biblical exposition.
This is true especially of the extensive sections given over to exhortation of
the recipients to be faithful and loyal to their confession (2:1-4; 3:7-4:13;
4:14-16; 5:11-6:12; 10:19-39; 12:1-29).
The biblical exposition gives the background and basis for such repeated
exhortations, but such exposition is not the author's basic interest and
purpose.

Moreover, chapter 13, with which we are particularly concerned, makes little use
of biblical exposition.
Only 13:6, with its quotation of Ps. 118:6, recalls the biblical background of
the writer's thinking and argument.
It is quite inadequate to describe Hebrews as biblical exposition.

(v) The most frequently used term to describe Hebrews is epistle or letter.
It contains essential truth.
The Greek word `epistle', ἐπιστολή, means a writing sent from a distance to
convey a message which the writer is not present to deliver directly by word of
mouth.
It reflects a living relationship between writer and recipient(s) and implies a
specific situation of sufficient importance to move the writer to send his
message so as to make it known to the recipients as soon as possible.
It expresses much of the truth to say that Hebrews is a letter.

Nevertheless, this statement needs to be qualified.
It is not a private personal letter, but is intended to be read to a group of
Christians met together for worship and consideration of their critical
situation.
It is intended to be a significant part of a service of fellowship and worship.
It therefore lacks something of the informal atmosphere normally found in a
personal letter from a friend.
Furthermore, the writer assumes without discussion that he has such a relation
to the Christians addressed that he can speak sternly to them with a note of
authority.
His writing is not just a friendly letter from an equal.
Written to play a prominent part in a service of worship and fellowship, and
written with a tacit assumption of authority, the writing is something more than
a private informal letter.

Many years ago Adolf Deissmann, impressed by the numerous letters found among
the ancient papyri discovered in Egypt, made a distinction between a letter, an
informal personal communication, and an epistle, written with a more formal note
and intended for public reading or for reading as aform of literature.
There is such a distinction as Deissmann pointed out.
But the use which Deissmann made of this distinction was that many of our New
Testament letters are real letters and not literary epistles.
To a degree Deissmann's distinction is well founded.
The New Testament writers of letters were not seeking literary recognition.
They did not write their letters for publication.
In almost al cases they wrote to meet the needs of a local church or a group of
churches; they wrote to speak to current situations and to meet present needs.

But every New Testament letter is much more than a private personal letter.
Each was written to be read to a group of Christians gathered together for
fellowship, worship, and consideration of their situation and duty.
The note of authority with which Paul wrote cannot be ignored, even in letters
which traditionally are considered personal letters.
Philemon was sent to merely to Philemon, but also to Apphia and Archippus `and
the church in your house' (Philemon 2).
Each of the Pastoral Letters, whatever their authorship, was written to the
Church in a rather general way; the word `you' in the closing prayer asking
God's grace for the recipient(s) is in the plural number (I Tim. 6:21; II Tim.
4:22; Titus 3:15).
III John comes as near to being a personal letter as any letter in the New
Testament, but even in this brief letter a more than private use of the letter
is probably taken for granted (cf. III John 15: `Greet the friends, every one of
them').

These observations help us to see that Hebrews with its somewhat formal attitude
to the Christians addressed and its rather formal literary structure in the
first twelve chapters of the writing is not so radically different from the
other New Testament `letters' as is sometimes thought.
In a sense it may be called a letter, for it was a written message sent to a
church (or possibly to a local group of churches) to be read to its assembly,
but it is a quite formal kind of letter, and only in the closing section of the
writing does the personal relation of the writer to the recipients emerge
clearly and in a somewhat more informal way.

(vi) Hebrews may be called an exhortation (παρἀκλησις).
Indeed, this is what the writer himself calls it, a `word of exhortation'
(13:22).
Large portions of the writing fit this description perfectly (2:1-4; 3:7-4:13;
4:14-16; 5:11-6:12; 10:19-39; 12:1-29; and the frequent imperatives and
exhortations of chapter 13).

The writer obviously has no interest in theological discussion for its own sake.
He is concerned to give the recipients a right view of Jesus Christ and his
aving work, in order to show how great a privilege the recipients have and what
an immense and irreparable loss they would suffer it they let the passage of
time, the hardships of discipleship, or the lure of any other loyalty rob them
of their joy in faith and faithfulness in life.
We understand Hebrews rightly only if we keep this urgent note of exhortation
clearly before us in all our discussion of the form and meaning of the writing.

How then can we best describe the literary form represented by Hebrews?
If it is true, as we have contended, that no one English word will clearly and
accurately express what kind of literary form Hebrews embodies, then we must
state our answer in the form of a sentence which successively focuses on a seris
of points, each of which is an integral part of the full answer.

A tentative answer would be as follows: Hebrews is a written message, which sets
forth vital aspects of the Christian gospel on the basis of Scripture (which to
the writer was of course the Old Testament); it was sent from a distane to be
read aloud to a Christian congregation assembled for worship, fellowship, and
instruction; and it was the work of a leader who was known to these Christians
and could speak to them concerning their current situation with a note of
authority and urgent exhortation and deep pastoral concern for their total
Christian life.

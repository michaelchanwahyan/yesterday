\section{The Literary Form of Hebrews}
The difficulty in identifying the literary form of Hebrews is that none of the
terms we use to describe modern types of literary works fits exactly the form
and nature of Hebrews.
We naturally try to describe such a writing by one English word.
But in so doing we are inevitably led to use an inexact term, and so we conceal
to some degree the actual nature of the writing we are studying.
It will further our study if we list the chief terms used to describe the
literary nature of Hebrews and note the truth and the misleading implications of
each term.

(i) Hebrews has been called an essay or treatise.
These terms point to the serious, orderly, scholarly treatment of a theme, and
to some extent Hebrews fits this description.
But they fail in at least two respects to represent the nature and literary form
of this writing.
For one thing, a treatise or essay is a general discussion of some aspect of
truth and life, but  Hebrews was directed to a definite group of Christians and
concerned their urgent life situation.
It was to be read aloud to hat group to help it meet its crisis.

Moreover, a treatise or essay is content-centred; it aims to clarify truth.
But Hebrews is marked by repeated and urgent exhortation directed to the special
group addressed.
This personal focus and this hortatory tone are not adequately expressed by such
terms as treatise or essay.
The author's main attention is directed to the life situation of the people
addressed.
The concern of the writer is to guide those addressed to act loyally and
responsibly in the face of that situation.

(ii) The word oration has been used, though rarely, to describe Hebrews.
This term takes account of the fact that Hebrews was written to be read aloud to
a definite group of Christians.
But in itself the word oration fails to express the basic fact that this writing
was prepared to be read to a group from whom the writer is separated at the
time.
The written document is to some extent a substitute for an oral message directly
spoken to the people whom the author would like to address in person.
Someone else must read this writing to the people addressed.
The word oration does not express this fact.

(iii) Hebrews has often been called a sermon or homily.
In many ways this term is useful and accurate.
It expresses the personal concern of the writer for the spiritual welfare and
faithful integrity of the people addressed.
It indicates that this is a biblically-based Christian message which has an
assembled congregation in mind (or, if it is to be read to more than one
congregation in the city of its destination, it has all of these assembled
congregations in mind).
In a sermon or homily the note of exhortation, of urgent appeal, is inevitably
present, as it is in Hebrews.

Yet in at least two respects the word sermon or homily is not a correct and
adequate description of Hebrews.
In the first place, a sermon properly speaking is the direct personal statement
of Christian truth and the exhortation to the hearers to heed this truth.
It is of the essence of a sermon that the preacher faces his congregation and
speaks in person what he believes is the urgent word of God to them at the time.
But Hebrews is sent from a distance to be read to a congregation in the absence
of the author.

Moreover, the widely ranging general imperatives inchapter 13 are not what we
expect in a sermon, which normally comes to a conclusion and climax in the clear
statement and application of the aspect of truth on which the sermon centres.
One does not expect a sermon to conclude with a miscellaneous medley of
commands, personal information, and general instructions.
Hebrews is something more than and different from a sermon or homily.

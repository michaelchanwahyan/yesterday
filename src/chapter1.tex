\chapter{INTRODUCTION}
Far too much of the study devoted to the `Letter to the Hebrews' has dealt with
questions whose answer eludes us.

Who wrote it ?
From ancient times a tradition has ascribed it to Paul, but from earliest days
there was no agreement as to authorship.
\footnote{ A detailed study of the tradition in the East and in the West is
found in Franz Overbect, {\it Zur Geschichte des Kanons} (Chemnitz: Ernst
Schmeitzner, 1880), pp. 1-70.}
In the Western Church, centred in Rome, the writing was known early, for
Clement of Rome, writing about AD 96, reflects knowledge of it,
\footnote{See e.g. the use of Hebrews I in {\it I Clem.} 36.}
but Pauline authorship was not accepted in the West until the fourth century.
\footnote{A concise summary of data on `Authorship' and `Canonicity' is given
by F. F. Bruce in {\it The Epistle to the Hebrews} (London: Marshall, Morgan
and Scott, and Grand Rapids: Wm. B. Eerdmans Publishing Co., 1964),
pp. xxxv-xlii, xliv-xlvii.}
Barnabas was declared to be the author by Tertullian
\footnote{In his work, {\it On Modesty (De pudicitia)}, ch. 20. A good
statement of the case for this view is given by Eduard Riggenbach, {\it Der
Brief an die Hebraer} (Leipzig: Deichert, 1913), pp. xxxiv-xlviii.};
Irenaeus, Hippolytus, and Gaius of Rome did not accept Pauline authorship; the
Muratorian Canon does not include it among Pauline authorship in the West.

The Eastern Church tended from early times to accept Pauline authorship, but
this view was not unchallenged, and when held it was subject at first to
interesting qualifications.
Clement of Alexandria held that Paul wrote it in Hebrew and Luke translated it
into its present Greek form.
\footnote{See Eusebius, {\it Church History}, VI.14.2-4.}
Origen, keenly aware of the fact that Hebrews differs from the unquestioned
Paauline lettrs, and acquainted with views that ascribed the writing to Clement
of Rome or Luke, concluded that the thoughts came from Paul, but he added that
`only God knows who actually wrote the letter'.
\footnote{The relevant passages are quoted in Eusebius, {\it op.cit.},
VI.25.11-14.}

By the end of the ourth century, Pauline authorship was generally accepted, and
this remained true until the Reformation.
Then Erasmus renewed the ancient doubts.
Luther conjectured that Apollos was the author, and other prominent Reformers,
Calvin included, could not accept Paul as the author.
\footnote{See Bruce {\it op. cit.}, pp. xxxix-xli, xlvi-xlvii.}
The question of authorship continues to be discussed, but apart from the Roman
Catholic view that Paul was the author and some disciple of his the actual
writer,
\footnote{The three decisions on this question given by the Pontifical
Biblical}
the general view today is that Paul was not the author.
\footnote{A prominent exception to this statement is William Leonard, {\it The
Authorship of the Epistle to the Hebrews} (London: Burns, Oates, and
Washbourne, 1939). He defends direct Pauline authorship.}
A few scholars accept the tradition that Barnabas wrote Hebrews, some think
that Apollos was the author, but a large number are content to say that the
author is unknown.

The reference to `our brother Timothy' in 13:23 may indicate that the author
had contacts with or belonged to the wing of the Church influenced by Paul, but
the remarkable differences from Paul in thought and style warn against making
too much of this possibility.
In any case, 2:3 indicates that the author wrote at the earliest in the
second generation of the Church.

To whom was Hebrews written ?
It may be suggested that the Christians addressed were Roman residents, and
that their past persecution mentioned in 10:32-34 was Nero's persecution of
the Christians at Rome about AD 64.
In that case the recipients were Roman Christians.
But 12:4 states that the church addressed had suffered no blood martyrdoms.
The only clue to the location of the Christians addressed is found in 13:24,
where {\it οὶἀπὀτῆς᾽ Ιταλίας} probably means `those who come from Italy',
Christians whose homeland is Italy but who at the time of writing are with the
writer in some other place and send greetings back to their Christian comrades
in Intaly.
In this case the church or church groups addressed most likely were located in
the city of Rome.
But possibly the words mean `those here with me in Italy' send greetings to
`you' (the recipients) in some other unnamed place.
In this case we can say nothing definite concerning the place of residence of
the recipients of Hebrews.

Where was Hebrews written ?
As just noted, the place of writing was just possibly Rome or some other place
in Italy, but probably was some other city whose identity eludes us.

When was Hebrews written ?
It often has been argued that it must have been written before the fall of
Jerusalem in AD 70, since after that date the writer would surely have
pointed to the fall of Jerusalem to support his case.
It has been contended that this event, which included the destruction of the
Temple and the cessation of the Jewish sacrifices there offered, would
inevitably have been mentioned by the author of Hebrews, if he wrote after AD
70; he would have used that event as proof that God had set aside the
outmoded Jewish sacrificial system.
But it has rightly been noted that the writer discusses the Jewish sacrificial
system solely on the basis of the Old Testament Scripture and the wilderness
`Tent' or tabernacle there described.
He does not refer to the Jerusalem temple.
It would therefore go too far to say that the writer of Hebrews would have been
bound to refer to the destruction of Jerusalem and the Jewsih Temple had he
written after AD 70.
The date of the writing of Hebrews cannot be determined with certainty; it
could be before AD 70, but it could with some reason be set not long before
AD 95.

It is unfortunate that so much attention has been paid to questions of
authorship, destination, place of writing and date.
No adequate evidence is available to support a definite and dependable answer.
The frustratingly inconclusive study of Hebrews should make it clear that we
cannot find certain answers to the questions: Who? To whom? From where? When?

In such a situation the one promising line of study open to us is an intensive
examination of the writing itself.
The athor of Hebrews was admittedly a careful and competent writer.
His own words can tell us much about himself, his purpose, and the situation of
the people he addressed.

The present study proceeds in a way which to our knowledge has not been used
before.
Its primary focus is on the more informal and personal material of chapter 13.
This concluding chapter presents critical questions which demand discussion.
At the same time, it contains statements and insights of great importance for
the understanding of the entire writing.
We first attempt, in ch. 2, to throw new light on the form and function of
chapter 13, and then in ch. 3 proceed to show by an examination of the main
themes of chapter 13 how far this concluding section is vitally linked with
the preceding chapters 1-12.

